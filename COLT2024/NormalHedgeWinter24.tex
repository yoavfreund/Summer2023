\documentclass[anon,12pt]{colt2024} % Anonymized submission
%\documentclass[final,12pt]{colt2024} % Include author names

% The following packages will be automatically loaded:
% amsmath, amssymb, natbib, graphicx, url, algorithm2e

\title[Second order NormalHedge]{Second order bounds for NormalHedge}
\usepackage{times}
% Use \Name{Author Name} to specify the name.
% If the surname contains spaces, enclose the surname
% in braces, e.g. \Name{John {Smith Jones}} similarly
% if the name has a "von" part, e.g \Name{Jane {de Winter}}.
% If the first letter in the forenames is a diacritic
% enclose the diacritic in braces, e.g. \Name{{\'E}louise Smith}

% Two authors with the same address
% \coltauthor{\Name{Author Name1} \Email{abc@sample.com}\and
%  \Name{Author Name2} \Email{xyz@sample.com}\\
%  \addr Address}

% Three or more authors with the same address:
% \coltauthor{\Name{Author Name1} \Email{an1@sample.com}\\
%  \Name{Author Name2} \Email{an2@sample.com}\\
%  \Name{Author Name3} \Email{an3@sample.com}\\
%  \addr Address}

% Authors with different addresses:
\coltauthor{%
 \Name{Author Name1} \Email{abc@sample.com}\\
 \addr Address 1
 \AND
 \Name{Author Name2} \Email{xyz@sample.com}\\
 \addr Address 2%
}

%\usepackage{fullpage}
% Packages
\usepackage{amsfonts}
\usepackage{amsmath}
%\usepackage{amsthm}
\usepackage{graphicx}

%\theoremstyle{plain}
\usepackage{amsmath,bm}

\newtheorem{lemma}{Lemma}
\newtheorem{claim}[lemma]{Claim}
\newtheorem{theorem}[lemma]{Theorem}
\newtheorem{corollary}[lemma]{Corollary}
\newtheorem{definition}{Definition}
\newtheorem{question}{Question}

\newcommand{\note}[1]{\par{\bf Note:}#1\par}
\newcommand{\notem}[1]{{\marginpar{\tiny #1}}}

\newcommand{\figline}{\rule{\textwidth}{1pt}}

%\newcommand{\proof}{\noindent{\bf Proof:} }
%\newcommand{\qed}{\rule{0.7em}{0.7em}}

\newcommand{\newmcommand}[2]{\newcommand{#1}{{\ifmmode {#2}\else\mbox{${#2}$}\fi}}}
\newcommand{\newmcommandi}[2]{\newcommand{#1}[1]{{\ifmmode {#2}\else\mbox{${#2}$}\fi}}}
\newcommand{\newmcommandii}[2]{\newcommand{#1}[2]{{\ifmmode {#2}\else\mbox{${#2}$}\fi}}}
\newcommand{\newmcommandiii}[2]{\newcommand{#1}[3]{{\ifmmode {#2}\else\mbox{${#2}$}\fi}}}

\newcommand{\E}[2]{{\bf E}_{#1}\left[ #2 \right]}
%\newcommand{\D}[2]{{\bf D}_{#1}\left[ #2 \right]}
\newcommand{\D}{{\cal D}}

\renewcommand{\P}[2]{{\bf P}_{#1}\left[ #2 \right]}
\newcommand{\reals}{\mathbb{R}}

\newcommand{\argmin}{\mbox{argmin}}
\newcommand{\argmax}{\mbox{argmax}}
\newcommand{\SP}[1]{{\cal P}^{#1}} % strictly positive function of
                                   % order i
\newcommand{\R}{R}      % Cumulative regret
\newcommand{\vR}{\mathbf{R}} %regret vector

\renewcommand{\r}{r}      % Instantaneuous regret
\newcommand{\state}{{\bf \Psi}}
\newcommand{\fstate}{{\bf \hat{\Psi}}}
\newcommand{\1}[1]{{\mathbf 1}\left[#1\right]} % point mass at #1

\newcommand{\pot}{\phi}
\newcommand{\potPQ}{\pot_{\learnerM,\adversM}}
\newcommand{\potLA}{\pot_{\l,\a}}
\newcommand{\finalPot}[1]{\pmb{\phi}_{#1}}
\newcommand{\finalPotT}{\finalPot{T}}
\newcommand{\finalPotR}{\finalPot{\realT}}
\newcommand{\finalPotTau}{\finalPot{\tau}}
\newcommand{\upperpot}{\pot_{\learnerM}^{\downarrow}}
\newcommand{\upperpotb}{\pot_{\learnerMb}^{\downarrow}}
\newcommand{\upperpotd}{\pot_{\learnerMd}^{\downarrow}}
%\newcommand{\upperpotj}{\pot_{\learnerM(j)}^{\downarrow}}
\newcommand{\upperpotMdk}{\pot_{\learnerMdk}^{\downarrow}}
\newcommand{\upperpotMdj}{\pot_{\learnerMdj}^{\downarrow}}

\newcommand{\lowerpot}{\pot_{\adversM}^{\uparrow}}
\newcommand{\lowerpotb}{\pot_{\adversMb}^{\uparrow}}
\newcommand{\lowerpotd}{\pot_{\adversMd}^{\uparrow}}
\newcommand{\lowerpotj}{\pot_{\adversM(j)}^{\uparrow}}
\newcommand{\lowerpotMdk}{\pot_{\adversMdk}^{\uparrow}}
\newcommand{\lowerpotMdkp}{\pot_{\adversMdkp}^{\uparrow}}
\newcommand{\lowerpotMdkpar}[1]{\pot_{\adversMdkpar{#1}}^{\uparrow}}
\newcommand{\lowerpoteven}{\lowerpot_{\evensplit}}

\newcommand{\realT}{\mathcal{T}}  % the final time for the real-time
% game
\newcommand{\Tset}[1]{\pmb{T}_{#1}}
\newcommand{\Ilat}[1]{\pmb{I}_{#1}}
\newcommand{\Klat}[1]{\pmb{K}_{#1}}
\newcommand{\score}{\Phi}
\newcommand{\upperscore}[1]{\score_{#1}^{\downarrow}}
\newcommand{\lowerscore}[1]{\score_{#1}^{\uparrow}}

\newcommand{\upperscoreM}{\upperscore{\learnerM}}
\newcommand{\upperscoreMd}{\upperscore{\learnerMd}}
\newcommand{\upperscoreMdk}{\upperscore{\learnerMdk}}

\newcommand{\lowerscoreM}{\lowerscore{\adversM}}
\newcommand{\lowerscoreMd}{\lowerscore{\adversMd}}
\newcommand{\learnerM}{P}
\newcommand{\learnerMb}{P_I}
\newcommand{\learnerMd}{P_D}
\renewcommand{\l}{\learnerM}
\newcommand{\learnerMdk}{P_{D(k)}}
\newcommand{\learnerMdj}{P_{D(j)}}
\newcommand{\legalLearner}{{\cal L}}

\newcommand{\adversM}{Q}
\newcommand{\adversMb}{Q_I}  %Integer time game
\newcommand{\adversMd}{Q_D}  % Discrete time game
\newcommand{\adversMdkpar}[1]{Q_{D(#1)}}
\newcommand{\adversMdk}{\adversMdkpar{k}}  %Discrete time game with step of
% size 2^{-k}

\newcommand{\adversMdkp}{\adversMdkpar{k+1}}  %\adversMdk with k+1

% \renewcommand{\a}{\adversM}
%\newcommand{\legalAdversary}{{\cal A}}
\newcommand{\agloss}{v}
\newcommand{\Bias}{B}
\newcommand{\deltat}{\Delta t}

\newcommand{\diffweight}{\l^d}   % learner strategy based on taking a difference
\newcommand{\upperpotdiff}{\upperpot_{\diffweight}}


\newcommand{\Binom}{\mathbb{B}}
\newcommand{\radsum}{\Binom(s_1,\ldots,s_T)}
\newcommand{\var}{\mbox{Var}}
\newcommand{\V}{V}


\newcommand{\at}[1]{\left\{ \left. #1
    \right|_{\begin{tiny}\begin{matrix}
          \tau,\rho=\\t_i,\R \end{matrix} \end{tiny}}
    \pot(\tau,\rho)\right\}}
\newcommand{\att}[1]{\left\{ \left. #1  
\right|_{\begin{tiny}\begin{matrix} \tau,\rho=\\t_i+g \deltat_i,\R_i+g
      r_i \end{matrix} \end{tiny}}
\pot(\tau, \rho)\right\}}


\newcommand{\atI}[1]{\left\{ \left. #1  
\right|_{\begin{tiny}\begin{matrix}
      x,y=\\x_0,y_0 \end{matrix} \end{tiny}}
f(x,y) \right\}}
\newcommand{\atII}[1]{\left\{ \left. #1
\right|_{\begin{tiny}\begin{matrix}
      x,y=\\x_0+t\Dx,y_0+t\Dy \end{matrix} \end{tiny}}
f(x,y) \right\}}


\newmcommandi{\paren}{\left({#1}\right)}
\newmcommandi{\brac}{\left[{#1}\right]}



%%% Local Variables:
%%% mode: latex
%%% TeX-master: "BrownianIsWorstCase"
%%% End:


\begin{document}

\maketitle
\begin{abstract}
  We study potential-based online learning algorithms in a game
  theoretical setting. We characterize the min/max optimal strategies
  for the set of potential functions that have strictly positive
  derivatives up to order four. We show that the optimal adversarial
  strategy for all of these cases corresponds to Brownian Motion.

  Based on this analysis we design a variant of NormalHedge which has
  second order upper bounds that are closely matched by lower-bounds.
  
\end{abstract}

\section{Outline}

Main results
\begin{enumerate}
\item Upper bound using variance for normal-hedge.
\item almost matching lower bound using variance on any online algorithm.
\end{enumerate}

Technical results:
\begin{enumerate}
\item A min/max analysis of a finite horizon potential game for
  terminal potential with 4 strictly positive derivatives.  Resulting
  in a characterization of the min/max strategies as solutions of the
  heat equation.
\item Using analytical solutions of the heat equation, we create a
  stop-anytime version of the learning algorithm.
\end{enumerate}

\subsection{Organization of paper}
\subsubsection{Main Resullts}
\begin{itemize}
\item Upper bound using variance for normal-hedge.
\item almost matching lower bound using variance on any online algorithm.
\end{itemize}
\subsubsection{The potential game}
\begin{itemize}
\item Define the generalized potential game: give range and allow arbitrary split.
\item The binomial strategies.
\item defining upper and lower potentials.
\item the limit range $\to 0$ is best for adversary.
\item the limit range $\to 0$ is min max.
\item is characterized by heat equation.
  \end{itemize}
\subsubsection{Variance based bounds}
\begin{itemize}
\item proof of variance based bound for any 4-convex function.
\item proof of variance bound for normal hedge and why cannot improve this potential.
\item proof of lower bound.
  \end{itemize}

\section{Introduction}
In this paper we study the {\em the decision-theoretic online learning game}
(DTOL)~\cite{freund1997decision}. DTOL (Figure~\ref{fig:DTOL}) is a
repeated zero sum game between a {\em learner} and an {\em
  adversary}. The adversary controls the losses of $N$ actions, while
the learner controls a distribution over the actions.
\newcommand{\Pvec}[1]{\vec{P}^{#1}}
\newcommand{\lvec}[1]{\vec{l}_{#1}}
\newcommand{\Lvec}[1]{\vec{L}_{#1}}
\newcommand{\Rvec}[1]{\vec{R}^{#1}}

\begin{figure}[ht!]
\framebox{
\begin{minipage}[t]{6.4in}
  {\bf Initialization:} $R^0_j=0$ for $j=1,\ldots,N$\\
  ~\\
For $i=0,1,\ldots$
\begin{enumerate}
\item The learner chooses a distribution function $P_j^i$ over the
  experts $j \in \{1,\ldots,N\}$ \\ such that $\sum_{j=1}^N P_j^i=1$. We
  denote the distribution by $\Pvec{i} = \{P_1^i,\ldots,P_N^i\}$.
\item The adversary chooses the {\em instantaneous loss for 
    expert}
  $j \in \{1,\ldots,N\}$: $l_j^i \in [-1,+1]$
\item The learner incurs an {\em instantaneous loss of the learner} is defined
  as $\ell^i = \sum_{s=1}^i P^i_j l_j^i $ 
\item The {\em cumulative regret} of the learner with respect to
  expert $j$ is \\$\R_j^{i+1} = \R_j^i +l^i_j-\ell^i$. 
  $\Rvec{i+1}$ represents the regret vector after iteration $i$.
\end{enumerate}
The goal of the algorithm is to minimize the maximal regret.
\end{minipage}}
\caption{Decision theoretic online learning \label{fig:DTOL}}
\end{figure}


\cite{freund1997decision} presents the DTOL framework and applies the
weighted majority algorithm~\cite{LittlestoneWa94} to define the {\em
  multiplicative weights algorithm} which yield {\em zero-order}
regret bounds of the form
\begin{equation} \label{eqn:0-order-bound}
  \max_{j=1}^N \R_j^T \prec \sqrt{T \ln N}
\end{equation}
We use $\prec$ to indicate that we ignore lower order terms. 

Many generalizations and refinements have been made over the years. We
describe those that are relevant to our paper below
\begin{itemize}
\item{\bf percentile-based bounds} \cite{freund1999adaptive} refines
  this bound by replacing the comparison to the best single expert to
  a comparison to the regret to the top $\epsilon$-percentile of the
  experts, indicated by $\R(T,\epsilon)$, giving bounds of the form
\begin{equation} \label{eqn:percentile-bounds}
\R(T,\epsilon) \prec \sqrt{T \ln \frac{1}{\epsilon}}
\end{equation}
Using quantiles allows experts sets that are uncountably infinite.
For example consider an expert set that corresponds to linear
functions with real valued parameters.

\item{\bf Parameter-free algorithms} The multiplicative weights
  algorithm has a learning rate parameter $\eta>0$. Achieving the
  bounds~\ref{eqn:0-order-bound} and~\ref{eqn:percentile-bounds}
  requires a priori-knowledge of $T$ and either $N$ or       
  $\epsilon$. ``Parameter-free'' algorithms remove the need for
  a-priori knowledge of the sequence. Parameter-free algorithms have
  been proposed in~\cite{chaudhuri2009parameter,
    chernov2010prediction,orabona2016coin,cutkosky2018black}.

\item{\bf Second order bounds} take advantage of so-called ``easy
  sequences'' where $|l_j^i|<1$. Roughly speaking, second order bounds
  replace the length of the sequence $T$ with a sum of quadratic terms
  of the form $(l_j^t)^2$ or $(l_j^t-\ell_t)^2$. Two types of second
  order bounds were proposed in~\cite{cesa2007improved}.  The {\em
    Quadratic Variation} $Q_j^T=\sum_{t=1}^T (l_j^t)^2$ measures the
  cumulative variation of the expert $j$ at. While the {\em cumulative
    variance}
  $V^T = \sum_{t=1}^T \sum_{j=1}^N H^t_j (l_j^t - \ell^t)^2$ is a
  measure of the variance of all of the experts.~\footnote{Note that, in our setting, $H_j^t$ can be different from $P_j^t$.} There are numerous
  publications that give bounds based on quadratic
  variation~\cite{chernov2009prediction,hazan2010extracting,gaillard2014second,koolen2015second}. But
  none that are based on cumulative variance. In this paper we give an
  upper and lower bounds based on cumulative variance. Thereby
  providing evidence that the cumultive variance provides a tighter
  characterization of ``easy sequences'' than quadratic variation.
  
\item {\bf Potential based Online learning algorithms} There are two
  main paradigms for designing online learning algorithms: the {\em
    follow the leader} paradigm and the {\em potential function}
  paradigm.  The second is the paradigm we study here.  
\end{itemize}

\subsection{Main Results}
In this paper we present a new analysis of the algorithm NormalHedge
presented in~\ref{sec:NormalHedgeAlg} and improved upon
in~\cite{luo2015achieving}.  NormalHedge is potential based and
parameter free, the bounds on its performance or of the
form(~\ref{eqn:percentile-bounds}).

\paragraph*{Second order regret bounds for NormalHedge}

\begin{definition}[weighted variances]
We use the definitions of $\{l^i_j\}_{j=1}^N$ and $\ell^i$ given in the DTOL game (Fig.~\ref{fig:DTOL}).
Let $H^i=(H^i_1, H^i_2,\ldots , H^i_N) $ be a distribution over experts at iteration $i$.
Let ${\cal H}=(H^1,H^2,\ldots)$ be a sequence of distributions.
We define the ${\cal H}$-weighted-variance as
\begin{equation}
  v^i(H^i)=\sum_{j=1}^N H^i_j (l_j^i-\ell^i)^2
\end{equation}
We define the ${\cal H}$-cumulative-variance to be
$$V^T({\cal H}) = \sum_{i=1}^T v^i(H^i)$$
\end{definition}

Our regret bounds relate the ${\cal H}$-cumulative-variance $V^T({\cal H})$,
the percentile of the comparison set of experts $\epsilon$ and the regret relative to the comparison set $R$.
For the sake of simplicity we upper and lower bound $\epsilon(V^T({\cal H}),R)$. These bounds are equivalent
to bounds on $R(V^T({\cal H}),\epsilon)$.
 
Our main result is that NormalHedge is a parameter free
stop-anytime algorithm with the following regret bound that holds simultanously for all $T,R$
\begin{theorem} \label{thm:NormalHedge}

Let $V^T$ be the ${\cal H}$-cumulative variance 
for ${\cal H}$ defined in line 3 of~(\ref{eqn:HNormalHedge})

then for all $T,R$ the Normal-Hedge algorithm, defined in Figure~\ref{fig:normalhedge2} satisfies

\begin{equation} \label{eqn:epsilonUpperBound}
  \epsilon(V^T({\cal H}),R) \leq \sqrt{V^T({\cal H})+1} \;\; \exp \paren{\frac{-R^2}{2(V^T({\cal H})+1)}}
\end{equation}
\end{theorem}

Our lower bound 
\begin{theorem} \label{thm:LowerBound}
  Given any sequence of numbers
  $0 \leq \nu^i \leq 1$ such that
  $\lim_{T \to \infty} \sum_{i=1}^T \nu_i = \infty$, there exists an
  adversarial strategy such that $V^T({\cal I}) = \sum_{i=1}^T \nu_i$
  for any sequence of distributions ${\cal I}$, and any learning
  algorithm, the following holds.

  For any $a>2$ define the sequence of regrets $R^T = a \sqrt{V^T({\cal I)}}$, then
  \begin{equation} \label{eqn:epsilonLowerBound}
    \lim_{T \to \infty}\epsilon(V^T({\cal I}),R^T) -
    \left(\frac{\sqrt{V^T({\cal I})}}{R^T} -  \left( \frac{\sqrt{V^T({\cal I})}}{R^T} \right)^3 \right) \exp \paren{\frac{-(R^T)^2}{2V^T({\cal I})}} \geq 0
  \end{equation}
\end{theorem}

If we define $a=\frac{R}{\sqrt{V^T({\cal H})}}$ we can approximate the upper bound by 
\[
\epsilon(a,R) < \approx a^{-1}R \exp\left(\frac{-a^2}{2} \right) 
\]
And the lower bound by
\[
  \epsilon(a,R) \geq (a^{-1}-a^{-3}) \exp\left(\frac{-a^2}{2} \right)
\]
If $a>2$ then the ratio between the upper and lower bound is $O(R)$.


Note that the ratio between the upper and lower bound on
$\epsilon(t,R)$ is, to first order, $R$. As the dependence of the
regret bound~\ref{eqn:upper-bound} on $\epsilon$ is a term of the form
$\log 1/\epsilon$ is we get that in the lower bound the term
$2\log \frac{1}{\epsilon}$ is replaced with
$2 \log \frac{1}{\epsilon R}$ which is a logarithmic gap between the
upper and lower bound on $R^t_\epsilon$.

\paragraph{Min max analysis of potential-based algorithms}
The lower bound~\ref{eqn:epsilonLowerBound} holds for any DTOL
algorithm and leaves a small gap with the upper bound
~\ref{eqn:epsilonUpperBound}. However, if we restrict ourselves to
potential-based online algrithms we can characterize the min/max
algorithm and adversary. This analysis holds for a broad set of
potential functions beyond the NormalHedge potential.  We formalize
the notion of potential based learning by defining a variant of the
DTOL game, where the goal of the learner is to minimize the average
potential. We derive the min-max solution of this game for any
potential function that has four strictly positive derivatives. In
particular, we show that the optimal strategy for the adversary
corresponds to Brownian Motion.

\subsection{Relation To Other Work}

TBD
% The analysis described here builds on a long line of work. Including
% the Binomial Weights algorithm and it's
% variants~\cite{cesa1996line,abernethy2006continuous,abernethy2008optimal}
% as well as drifting games~\cite{schapire2001drifting,freund2002drifting}.

\subsection{Organization of the paper}
The rest of the paper is organized as follows. In
Section~\ref{sec:NormalHedgeAlg}, we describe the algorithm that
satisfies Theorem~\ref{thm:NHUpperBound}. The rest of the paper is
devoted to the analysis of this and related algorithms.

In Section~\ref{sec:preliminaries} We define some terms and
notation. In Section~\ref{sec:PotentialGame} we define the potential
game which captures the structure of online learning algorithms that
have an and has an a-priori known end (or horizon) and a fixed
potential function for the end.  Having a known horizon and known
final potential function is needed for our analysis. Ultimately the
bounded horizon assumption is removed to allow algorithms, such as
NormalHedge, that can be stopped at any iteration.

We formalize the notion of the {\em state} of the game $\state$, which
allows us to separately analyze each iteration. In
Section~\cite{sec:potentialRecursion} we show how, for a fixed pair of
learner and adversarial strategies, we can extend the potential at the
horizon to a potential for any earlier iteration.

In Section~\ref{sec:disc-game-strategies} we describe strategies for the adversary and for the learner.


\begin{figure}[ht!]
\framebox{
\begin{minipage}[t]{6.4in}
Set $V^0=1$ \\
For $i=0,1,2,\ldots$ adversary decides when to stop

\begin{enumerate}
\item The learner chooses the distribution
\begin{equation} \label{eqn:learner-strat-cc}
  P^i_j =  \frac{1}{Z^i} \pospart{R_j^i \exp\paren{\frac{(R_j^i)^2}{2V^i}}}{R_j^i \geq 0}
  \mbox{ where } Z^i = \sum_{j=1}^N \pospart{R_j^i \exp\paren{\frac{(R_j^i)^2}{2V^i}}}{R_j^i \geq 0}
\end{equation}
\item The adverssary chooses the losses $|l^i_j|\leq 1$
\item The earner's loss is calculated:
  \begin{equation} %\label{eqn:ell-discrete}
    \ell^i=\sum_{j=1}^N P_j^i l_j^i,
  \end{equation}
  
\item  Update cumulative variance $V^{i+1}=V^i +v^i$ where
\begin{equation} \label{eqn:deltat}
  v^i=  \sum_{j=1}^N H(V^i,\R_j^i)(l_j^i-\ell^i)^2
\end{equation}
and 
\begin{eqnarray}  \label{eqn:HNormalHedge}
  W(V,\R) =
  \pospart{
    \exp\paren{\frac{R^2}{2V}} \left(\frac{1}{2V^{3/2}} + \frac{R^2}{V^{5/2}} \right)}
  {R\geq 0},
  H(V,\R) =\frac{W(V,R+1)}{\sum_{j=1}^N W(V+1,R_j-1)}
%              \frac{\sum_{j=1}^N W(V,R_j+1)}{\sum_{j=1}^N W(V+1,R_j-1)}
\end{eqnarray}

\item Update regrets $\forall j=1,\ldots N,\;\;\; R_j^{i+1} = R_j^i -l_j^t +\ell^i$

\end{enumerate}
\end{minipage}}
\caption{NormalHedge.2 \label{fig:normalhedge2}}
\end{figure}

\section{Preliminaries}

A few definitions and notations used in this paper:
\begin{definition}
A function $f:\reals \to \reals$ is strictly positive of degree $k$, 
denoted $f \in \SP{k}$ if the derivatives of orders 0 to $k$:  
$f(x), \frac{d}{dx}f(x), \ldots, \frac{d^k}{dx^k}f(x)$ exist and are strictly positive.
\end{definition}

Let $\vec{x}=(x_1,\ldots,x_n)$, and $f:R \to \R$, we define the {\em convolution} symbol $\odot$ as follows:
$$\vec{x} \odot f \doteq \frac{1}{n}\sum_{i=1}^{n} f(x_i)$$

Let $f : \reals \to \reals$ then
\[\pospart{f(x)}{x\geq 0} \doteq
  \begin{cases}
    f(x) & \mbox{ if } x \geq 0\\
    0    & \mbox{ otherwise }
  \end{cases}
\]

Let ${\cal N}(\mu,\sigma^2)$ be the normal distribution with mean $\mu$ and variance $\sigma^2$.

\newcommand{\finalpot}{\pot^{\mbox{\tiny final}}}

\section{The NormalHedge  Algorithm} \label{sec:NormalHedgeAlg}
Figure~\ref{fig:normalhedge2} describes a variant of the NormalHedge
algorithm~\cite{chaudhuri2009parameter} which satisfies the regret
bound given in Theorem~\ref{thm:NormalHedge}.

% In fact, the original algorithm can be used. The only difference is in
% the way the time (here - the cumulative variance) is updated. While
% the original version requires solving a system of two non-linear
% equations with two unknowns, the method here is simple averaging.
% This simpler update lends itself to the second order analysis which is
% obscured in~\cite{chaudhuri2009parameter}.



\begin{figure}[ht!]
\framebox{
\begin{minipage}[t]{6.4in}
  {\bf Initialization:} set initial regrets $R^0_j\geq 0$ for $j=1,\ldots,N$, a natural number $S$ \\
  and a final potential $\finalpot(\cdot) \in \SP{4}$, horizon $\realT>0$, initial time $t^0=0$
~\\
  
  For $i=0,1,\ldots$:
  \begin{enumerate}
   \item if $t^i=\realT$, let $T=i$ and exit loop
  \item The number of experts is $M^i=N 2^i$, They are indexed $1,\ldots,M^i$
  \item The adversary chooses a non-negative integer $k^i$ which defined the
    step size $s^i = 2^{-k_i} \leq \sqrt{\realT-t^i}$
  \item Time is updated $t^{i+1} = t^i + (s^i)^2$.
  \item The learner chooses a distribution function $P_j^i$ over the experts
  $j \in \{1,\ldots,M^i\}$ \\
  such that $\sum_{j=1}^{M^i} P_j^i=1$. We
  denote the distribution by $\Pvec{i} =  \{P_1^i,\ldots,P_{M^i}^i\}$.
\item The adversary splits each expert into two equal parts and assigns a loss to each part.\\
For $j \in \{1,\ldots,M^i\}$: $l_{j1}^i, l_{j2}^i,\in [-s^i,+s^i]$
\item The {\em instantaneous loss of the learner} is
  $\ell^i = \sum_{j=1}^i P^i_j \frac{l_{j1}^i + l_{j2}^i}{2}$. The
  adversary is restricted to chosing the losses so that $|\ell^i|\leq (s^i)^2$.
\item The {\em cumulative regret} of the learner with respect to
  expert $j1$ is $\R_{j1}^{i+1} = R_{j1} +l_{j1}^i -\ell^i$,\\
  and for expert $j2$ is $\R_{j2}^{i+1} = R_{j2} +l_{j2}^i -\ell^i$,\\
  $\Rvec{i+1}$ is regret vector at iteration $i+1$, it has $2M^i$
  components. %in a non-decreasing order.
   % $\vec{L}^{i+1}$ is the vector whose components are
   % $\{L_{11},L_{12},\ldots,L_{M^i1},L_{M^i2}\}$ ordered in as the
   % corresponding elements of $\Rvec{i+1}$
\end{enumerate}
The goal of the algorithm is to minimize the final score
$\score^T=\frac{1}{M^T}\sum_{j=1}^{M^T} \finalpot(R_j^T) = \vec{R}^T \odot \finalpot$
\end{minipage}
}
\caption{The potential game \label{fig:PotentialGame}}
\end{figure}

\section{The potential game}
In order to prove upper bounds on the regret we focus on
potential-based learning algorithms. 
We define a new game which we call
the {\em potential game}~ (Fig.\ref{fig:PotentialGame}).


Consider potential based algorithm for DTOL in which the horizon is pre-defined.

The potential game is similar to DTOL, with the following differences:
\begin{enumerate}
\item In DTOL, the range of the losses is always $[-1,+1]$, in the
  potential game the adversary sets the range of losses on iteration $i$ to $[-s^i,+s^i]$
  for $s^i = 2^{-2k_i}$ where $k_i$ is a natural number.
\item The goal in DTOL is to bound the maximal regret (or regret wrt the top $\epsilon$-percentile) at time $T$.
  the goal in the potential game is to minimize the final average potential.
\item In the potential game the horizon is a real value $\realT>0$
  defined in advance and the game ends when $\sum_i (s^i)^2 = \realT$
\item In the potential game, a pre-specified potential function
  $\finalpot(\cdot)$ is evaluated when the game ends.
\item At each iteration, the adversary splits each expert into two
  experts with equal weight and and assigns to each a loss in the
  range $[-s^i,s^i]$.
\end{enumerate}


***************** Till Here  *********************


\newcommand{\Ndist}{{\cal N}}

\subsection{Upper bound on Potential game implies regret bound on DTOL}

We show that an upper bound on $\score^T$ in the potential game
implies upper bounds on regret relative to the $\epsilon$ percentile in the bounded horizon DTOL.

\begin{theorem}
  Let $\pot^T$ be a strictly convex function of degree 2.  If a
  learner algorithm guarantees $\Theta > \score^T$ in the potential
  game, then the same algorithm guarantees
  $\epsilon(R) \leq \frac{\theta}{\finalpot(R)}$ in DTOL$(T)$ for all
  $R$
\end{theorem}
\begin{proof}
  We show that the adversary in the potential game is stronger than the adversary in
  DTOL. We do that by showing that any adversarial strategy for DTOL is also a valid strategy for the potential game.

  We show that the adversary for the potential game can simulate any
  adversary for DTOL.  First, we have the adversary set $k^i=s^i=1$
  for all iterations $i$, and $\realT = T$.  As a result, the number
  of game iterations is $T$, and $s^i = (s^i)^2 = 1$ for all $i$.
  Therefor also $|\ell^i|\leq 1$ holds because $|l^i_j| \leq s^i = 1$.

  Assume that an in the DTOL game expert $j$ suffers losses
  $l_j^1,l_j^2,\ldots$. In the potential game expert $j$ has $2^i$
  sub-experts on step $i$. We collapse all of these experts to a
  single expert by these experts the loses $l^i_{k1}=l^i_{k2}=l^i_j$

  A learner strategy that guarantees an upper bound
  $\theta\geq\score^T$ guaranteees the same upper bound for
  DTOL$(T)$. Finally we use the standard relationship between a
  potential that is convex of degree 2 and the regret.  Let
  $\epsilon(R)$ be the fraction of the coordinates of $\vec{R}^T$ that
  are larger or equal to $R$. Then
  $\theta \geq \score^T \geq \epsilon(R) \finalpot(R)$ which implies
  the statement of the theorem.
\end{proof}

\subsection{Strategies and Potentials}
\newcommand{\CA}{$\cal A$~}
\newcommand{\CL}{$\cal L$~}

Keeping the step size $s^i$ fixed, we describe strategies and
for the the adversary and for the learner, denoted \CA and \CL respectively. We focus on a single iteration $i$ and allow an arbitrary
$\vec{R}^i$ 

The strategy \CA is very simple: on iteration $i$ the losses for the two parts are set to
$l^i_{j1}=+s^i$, $l^i_{j2}=-s^i$. It is easy to see that if $\vec{R}^i=\{R^i_1,\ldots R^i_M\}$
then $\vec{R}^{i+1}=\{R^i_1-s^i,R_1^i+s^i,\ldots R^i_M-s^i,R^i_M+s^i\}$ regardless of the learner
choice of $\vec{P}^i$.

For \CA, and given the final potential $\finalpot$ we use
backward induction to associate a {\em lower potential} $\lowerpot^i$ with
iterations $T-1,T-2,\ldots,1$. The base step is $\lowerpot^T =
\finalpot$. Fixing $\lowerpot^{i+1}$, we define
\begin{equation} \label{eqn:lowerpot}
\forall x \in \reals, \lowerpot^i(x) = \frac{\lowerpot^{i+1}(x-s^i)+\lowerpot^{i+1}(x+s^i)}{2}
\end{equation}
  

Assuming that the adversry uses \CA in all iterations of the game it is easy to show that
\begin{equation} \label{eqn:potential-equality} 
  \vec{R}^0 \odot \lowerpot^0 = \vec{R}^1 \odot \lowerpot^1 = \cdots = \vec{R}^T \odot \lowerpot^T = \vec{R}^T \odot \finalpot = \score^T
\end{equation}
%As $\vec{R}^0=\{0\}$, $\vec{R}^0 \odot \lowerpot^0 = \lowerpot^0(0)$. In addition, the distribution of
%$\vec{R}$ is a binomial scaled by $s$. Combining these facts we get
%$$\lowerpot^0(0) = \vec{R}^T \odot \finalpot = \sum_{j=0}^T {T \choose j} \finalpot((2j-T)s)$$

We now move to the learner's strategy \CL. We use backward iteration to define
the lower potential and the learner's strategy.

The base case is $\upperpot^T=\finalpot$.

Given $\upperpot^{i+1}$ We define $\upperpot^i$ as follows
\begin{equation}
\forall x \in \reals, \upperpot^i(x) = \frac{\upperpot^{i+1}(x+s^i(1+s^i))+\upperpot^{i+1}(x-s^i(1+s^i))}{2}
\end{equation}
The strategy \CA is to set the learner's distribution function to
\begin{equation} \label{eqn:Pij}
P^i_j = \frac{1}{Z^i}(\upperpot^{i+1}(R_j^i+s^i(1+s^i)) - \upperpot^{i+1}(R_j^i-s^i(1+s^i)))
\end{equation}
where $Z^i$ is a normalization factor defined so that $\sum_j P^i_j =1$.

Equation~(\ref{eqn:potential-equality}) shows that \CA gives a lower bound (in fact, an equality) 
for the final score for any learner. We now show that
the \CL guarantees an upper bound on the final score for any adversary.
\begin{theorem}
  If the learner uses the strategy \CL on all iterations of the potential game then, for any choices made by the adversary
  \begin{equation} \label{eqn:potential-upper} \vec{R}^0 \odot
    \upperpot^0 \geq \vec{R}^1 \odot \upperpot^1 \geq \cdots \geq
    \vec{R}^T \odot \upperpot^T = \vec{R}^T \odot \finalpot = \score^T
\end{equation}
\end{theorem}
\begin{proof}
The proof is done by backward induction from $i=T$ to $i=0$. The base case is by definition
$\vec{R}^T \odot \upperpot^T = \vec{R}^T \odot \finalpot = \score^T$

For general $i$, show that $\vec{R}^i \odot \pot^i \geq \vec{R}^{i+1} \odot \pot^{i+1}$.
As $\pot^{i+1}$ is convex we have that for all $j,k$, $-s^i-(s^i)^2 \leq l^i_{jk}-\ell^i \leq s^i +(s^i)^2$
\begin{eqnarray} 
  \upperpot^{i+1}(R^i_{j}+l^i_{jk}-\ell^i)
  & \leq & \frac{\upperpot^{i+1}(R^i_j+s^i(1+s^i))+\upperpot^{i+1}(R^i_j-s^i(1+s^i))}{2} \label{eqn:upper-convex}\\
  &+&
      \left(l^i_{jk} - \ell^i\right) \frac{\upperpot^{i+1}(R^i_j+s^i(1+s^i))-\upperpot^{i+1}(R^i_j-s^i(1+s^i))}{2} \nonumber \\
  &=& \frac{\upperpot^{i+1}(R^i_j+s^i(1+s^i))+\upperpot^{i+1}(R^i_j-s^i(1+s^i))}{2} + \frac{\left(l^i_{jk} - \ell^i\right) Z^i P^i_j}{2} \label{eqn:using_def_upperpot}
\end{eqnarray}
Where~(\ref{eqn:upper-convex}) follows from convexity
and~(\ref{eqn:using_def_upperpot}) follows from the definition of
$P^i_j$ in Eqn.~(\ref{eqn:Pij})
Summing over the experts at iteration $i+1$ we get 
\begin{eqnarray} \label{eqn:upper-aver}
  \lefteqn{ \vec{R}^{i+1} \odot \upperpot^{i+1} = \sum_{j=1}^{M^i} \sum_{k=1}^2
  \upperpot^{i+1}(R^i_{j}+l^i_{jk}-\ell^i)}
  && \label{eqn:upperpot1}\\
  & \leq &  \sum_{j=1}^{M^i} \frac{\upperpot^{i+1}(R^i_j+s^i(1+s^i))+\upperpot^{i+1}(R^i_j-s^i(1+s^i))}{2} \label{eqn:upperpot2} \\
  & + & \sum_{j=1}^{M^i} \frac{\left(l^i_{j1} -\ell^i +l^i_{j2}- \ell^i\right) Z^i P^i_j}{2} \label{eqn:upperpot3}\\
  & = & \sum_{j=1}^{M^i}
        \frac{\upperpot^{i+1}(R^i_j+s^i(1+s^i))+\upperpot^{i+1}(R^i_j-s^i(1+s^i))}{2} \label{eqn:upperpot4} \\
        & = & \sum_{j=1}^{M^i} \pot^i(R_j^i) = \vec{R}^i \odot \pot^i \label{eqn:upperpot5}\\
\end{eqnarray}
Line (\ref{eqn:upperpot2}) follows from line (\ref{eqn:upperpot1}) and
Eqn. (\ref{eqn:using_def_upperpot}).
Line (\ref{eqn:upperpot4}) follows from lines (\ref{eqn:upperpot2})
and (\ref{eqn:upperpot3}) because the expression in line
(\ref{eqn:upperpot3}) is equal to zero, which follows from the plugging in the definition of $\ell^i$ (line 4. of Figure~\ref{fig:PotentialGame})
`\[
  \sum_{j=1}^{M^i} \left(l^i_{j1} +l^i_{j2}- 2\ell^i\right) P^i_j  =  \sum_{j=1}^{M^i} \left(l^i_{j1} +l^i_{j2}\right)P^i_j
  - 2 \ell^i = \sum_{j=1}^{M^i} \left(l^i_{j1} +l^i_{j2}\right)P^i_j - \sum_{j=1}^{M^i} \left(l^i_{j1} +l^i_{j2}\right)P^i_j
=0  \]
\end{proof}

Note that the upper and lower potentials $\lowerpot^i,\upperpot^i$
depend only on the final potential $\finalpot$ and on the iteration
number $i$. In particular they do not depend on the actions of the
learner (for $\lowerpot$) or the adversary (for
$\upperpot$). Moreover, the upper and lower bounds on the same depend
only on the convolutions
$\vec{R}^i \odot \lowerpot^i, \vec{R}^i \odot \upperpot^i$. This means
that, as far as our upper and lower bounds are concerned, all of the
information relevant to future of the game beyond iteration $i$ is in
the regret vector $\vec{R}^i$. This suggests calling $R^i$ the {\em
  state} of the game. Which, similarly to the state of a chess board,
contains all of the relevant information and makes past states
irrelevant.

\subsection{The adversary prefers smaller steps} \label{sec:smallsteps}

Equation~(\ref{eqn:lowerpot}) defines $\lowerpot^{i}$ in terms of $\lowerpot^{i+1}$ and $s_i$.
To better understand the choice of $s_i$ we re-index the iteration count $i$ with the cumulative time
$t^i = \sum{k=0}^i (s^k)^2$. We make the corresponding change in notation : $\lowerpot^i(R) = \lowerpot(t^i,R)$.
We now consider the question ``how should the adversary choose $s^i$?''

We phrase the question as follows: given two times $t^a$ and $t^b$,
and the potential function $\lowerpot(t^b,R)$, what is the sequence of
step sizes $s^i$ such that $\sum_i (s^i)^2 = t^b-t^a$ that would maximize $\lowerpot(t^a,R)$

To simplify the argument (which is generalized in the appendix)
we fix $t^a=0$,$t^b=\realT$,$\lowerpot(t^b,R) \in \SP{4}$ and compare $\lowerpot(t^a,R)$ for $s^0=1$ to
$s^0=s^1=s^2=s^3=1/2$

for the case $s=1$, the recursion is simple:
\begin{equation}
  \lowerpot^{1}(0,R) = \frac{\lowerpot(1,R-1)+\lowerpot(1,R+1)}{2}
\end{equation}
In the case $s=1/2$, $\Delta t = 1/4$ so there are four iterations, unfolding the recurion for these four steps we get
\begin{equation}
  \lowerpot^{1/2}(0,R) = \frac{1}{16}
  \left(\lowerpot(1,R-2s) + 4  \lowerpot(1,R-s)+ 6 \lowerpot(1,R)
  +4 \lowerpot(1,R+s) + \lowerpot(1,R+2s) \right)
\end{equation}
To see which one is larger we consider the difference:
\begin{equation}
  \lowerpot^{1/2}(0,R) -\lowerpot^{1}(0,R) = \frac{1}{16}
  \left(\lowerpot(1,R-2s) - 4  \lowerpot(1,R-s)+ 6 \lowerpot(1,R)
  -4 \lowerpot(1,R+s) + \lowerpot(1,R+2s) \right)
\end{equation}

As it turns out this expression is positive if $\finalpot \in \SP{4}$

Theorem~\ref{eqn:conergence-for-k} characterizes the limits of the
upper and lower potentials, as $k \to \infty$ are equal to each
other and to ${\cal N}(\R_0,\realT-t) \odot \finalPotR$ To show
that this limit corresponds to the min/max slolution of the game
we need to show that the adversary perfers smaller steps. In other
words, that for any $t,R$, $\lowerpotMdk(t,\R)$ increases with
$k$.

To prove this claim we strengthen the condition $\finalPotR \in \SP{2}$ used above to $\finalPotR \in \SP{4}$. In words, we assume that the function $\finalPotR(\R)$ is continuous and strictly positive and it's first four derivatives are continuous and strictly positive.

We use the sequence of discrete adversarial strategies
$\adversMdk, k=1,2,\ldots$ defined in
Section~\ref{sec:disc-game-strategies}.

\begin{theorem}\label{thm:smallerSteps}
  If $ \finalPotR \in \SP{4}$, and $\realT>0$  
  then for any $k\geq 0$, any $t \in [0,\realT]$ and any $\R$
  $$\lowerpotMdkpar{k+1}(t,\R) >  \lowerpotMdkpar{k}(t,\R)$$
\end{theorem}

The proof of the theorem relies on a reduction to a simpler case:
dividing a single time step of duration $\tau$ into four time steps of duration  $\tau/4$

\begin{lemma} \label{lemma:half-step}
  If $ \finalPotTau \in \SP4$, and $\tau>0$ then for any $\R$
  $$\lowerpotMdkpar{1}(0,\R) >  \lowerpotMdkpar{0}(0,\R)$$
\end{lemma}
  
\begin{proof} The step sizes for $k=0,1$ are
$s_0=\sqrt{\tau}, s_1=\frac{\sqrt{\tau}}{2}$ and the time icrements
are $\Delta t_0=\tau, \Delta t_1=\frac{\tau}{4}$. 
In other words $k=0$ corresponds to a single step of size $\sqrt{\tau}$, while $k=1$ corresponds
to {\em four} steps of size $\frac{\sqrt{\tau}}{2}$.

By definition $\finalPotTau(R)=\lowerpotMdkpar{0}(\tau,\R)=\lowerpotMdkpar{1}(\tau,\R)$

For $k=0$ we we get the recursion
\begin{equation}  \label{eqn:pot-recursion-0}
  \lowerpotMdkpar{0}(0, \R) =
  \frac{\lowerpotMdkpar{0}(\tau,\R-\sqrt{\tau})+
    \lowerpotMdkpar{0}(\tau,\R+\sqrt{\tau})}{2}
  =   \frac{\finalPotTau(\R-\sqrt{\tau})+
    \finalPotTau(\R+\sqrt{\tau})}{2}
\end{equation}

For $k=1$ we we have for $i=0,1,2,3$:
\begin{equation}   \label{eqn:lowerpotquarterstep}
   \lowerpotMdkpar{0}\paren{\frac{i}{4}\tau,\R}=
 \frac{\lowerpotMdkpar{0}\paren{\frac{i+1}{4}\tau,\R-\frac{1}{2}\sqrt{\tau}}+
   \lowerpotMdkpar{0}\paren{\frac{i+1}{4}\tau,\R+\frac{1}{2}\sqrt{\tau}}}{2}
 \end{equation}


Combining Equation~(\ref{eqn:lowerpotquarterstep}) for $k=0,1,2,3$ we get
\small
\begin{eqnarray} \label{eqn:pot-recursion-1}
  \lowerpotMdkpar{1}(0, \R)
  &=& \frac{1}{16}
      \left[\iter{-2}+4\iter{-} \right. \\
  &&+ \left. 6\iterzero +4\iter{+}+\iter{+2}\right] \nonumber\\
  &=& \frac{1}{16}
      \left[\fIter{-2}+4\fIter{-}+6\fIterzero +4\fIter{+}+\fIter{+2}\right] \nonumber
\end{eqnarray}
\normalsize the difference between
Equations~(\ref{eqn:pot-recursion-1}) and~(\ref{eqn:pot-recursion-0})
is \small
\begin{eqnarray} \label{eqn:pot-recursion-diff}
  \lefteqn{\lowerpotMdkpar{1}(0, \R) - \lowerpotMdkpar{0}(0, \R)}\\
&=&  \frac{1}{16}
\left[\fIter{-2}-4\fIter{-}+6\fIterzero -4\fIter{+}+\fIter{+2}\right] \nonumber
\end{eqnarray}
\normalsize Our goal is to show that the LHS of
Eqn.~\ref{eqn:pot-recursion-diff} is positive. This is equivalent to
proving positivity of
\begin{eqnarray}
g_a(\R) &=& \frac{2}{3a^2}\paren{\lowerpotMdkpar{1}(0, \R) -
  \lowerpotMdkpar{0}(0, \R)}\nonumber \\
& = &
\frac{1}{24 a^4}
      \left[\gIter{-2}-4\gIter{-}+6\gIterzero -4\gIter{+}+\gIter{+2}\right]
       \label{eqn:recursion-as-difference}
\end{eqnarray}
where $a=2 s_1=\sqrt{\tau}$
\end{proof}
The function $g_a(\R)$ has a special form called ``divided
differences''. The proof of the following lemma uses this fact to show that 
Eqn~(\ref{eqn:recursion-as-difference}) is strictly positive.
\begin{lemma} \label{lemma:divdiff}
If $\finalPotTau \in \SP{4}$ and $\tau>0$, then $\forall \R, g_a(\R)>0$
\end{lemma}
The proof of Lemma~\ref{lemma:divdiff} is given in appendix~\ref{sec:divdiff}

\begin{proof}  of Theorem~\ref{thm:smallerSteps} \\

Let $ \Klat{{\realT,k}}$ be the game lattices defined in
Equation~\ref{eqn:game-lattice}. The statement of
Lemma~\ref{lemma:half-step} can be trivially generalized as follows. Fix $k\geq 0$ and $t=i s_k^2$
then 
\begin{equation} \label{eqn:generalized-half-step}
  \mbox{ if } \phi((i+1) s_k^2,\R) \in \SP{4} \mbox{ then }
  \lowerpotMdkpar{k+1}(i s_k^2,\R) >  \lowerpotMdkpar{k}(i s_k^2,\R)
\end{equation}
The proof of the theorem is by double induction, first on $k=0,1,2,\ldots$ and then on
$t=\tau-s_k^2,\tau-2s_k^2,\ldots,0$.

The base case $k=0,t=\tau$ follows directly from Lemma ~\ref{lemma:half-step}.

For $k>0$ we use a second induction $t=\tau-s_{k-1}^2,\tau-2s_{k-1}^2,\ldots,0$, which, combined with 
~\ref{eqn:generalized-half-step} holds for $k-1$, shows that for any $(t,\R) \in \Klat{{\realT,k-1}}$
we have $ \lowerpotMdkpar{k}(t,\R) >  \lowerpotMdkpar{k-1}(t,\R)$

% we take a finite backward induction over
% $t=T-2^{-2k},T-2 \times 2^{-2k},T-3 \times 2^{-2k},\cdots,0$.
% Our inductive claims are that $\pot_{k+1}(t,\R) > \pot_{k}(t,\R)$ and
% $\pot_{k+1}(t,\R)$,$\pot_{k}(t,\R)$ are in $SP{4}$. That these claims carry over
% from $t=T-i \times 2^{-2k}$ to  $t=T-(i+1) \times 2^{-2k}$ follows
% directly from Lemma~\ref{lemma:n-strictly-convex}.

The theorem follows by forward induction on $k$.
\end{proof}

Combining Theorems~\ref{thm:seq-of-adv-strategies}
and~\ref{thm:smallerSteps} we get our characterization of the min/max
solution for the potential game.

\begin{theorem} \label{thm:min-max-limit}
Let $\realT>0$ and $\finalPotR \in \SP{4}$, define $\phi^*(t,\R) = {\cal N}(\R_0,\realT-t) \odot \finalPotR$
Fix a state at time $t<\realT$ $\state(t)$ and define the score of this state
to be $\score(t) = \state(t) \odot \pot(t)$. Then for any $\epsilon>0$ there exists
\begin{itemize}
\item A learner strategy that guarantees $\score^T \odot \finalPotR \leq \score(t)+\epsilon$ against any adversarial strategy and
  \item an adversarial strategy that guarantees  $\score^T \odot \finalPotR \geq \score(t)-\epsilon$
\end{itemize}
\end{theorem}
\begin{proof}
From Theorem~\ref{thm:seq-of-adv-strategies} we know that the  $k \to \infty$.
From Theorem~\ref{thm:smallerSteps} we know that the small step limit maximizes the 

Together these theorems show that the the min/max optimal potential function is $ {\cal N}(\R_0,\realT-t) \odot \finalPotR$.
\end{proof}

There seems to be a paradox: the adversary prefers to set $s_i$ as
small as possible but larger than zero. But there is no smallest positive (real) number,
whatever values the adversary chooses for $s^i$ it regrets not choosing a smaller value!

Luckily, this paradox does not bother the learner. The learner wants
to protect itself from an adversary with an arbitrarily small step
size it therefor uses the limit $s \to 0$ of the recursion.

\section{Characterizing the limit $s \to 0$}
\label{sec:continuous}
 We now charactrize this limit  $k \to \infty, s_k \to 0$.

%in both the lower bound
%(Eqn~(\ref{eqn:backward-iteration-lower-discrete})) or upper bound
%potential:
%(Eqn~(\ref{eqn:backward-iteration-upper-recursion-discrete})).

Consider the recursive equation for the lower potential:
 \begin{equation}
   \pot(t, \R) =  \frac{\pot(t+s^2,\R+s) + \pot(t+s^2,\R-s)}{2}
 \end{equation}

 By subtracting $\pot(t+s^2,R)$ from both sides and dividing by $s^2$ we get 
 \begin{equation}
   \frac{\pot(t, \R)  - \pot(t+s^2,R)}{s^2} =  \frac{\pot(t+s^2,\R+s) - 2\pot(t+s^2,R)+\pot(t+s^2,\R-s)}{2s^2} \label{eqn:difference}
 \end{equation}
 Taking the limit $s \to 0$ on both sides, we get, from the definition of first and second order partial derivatives, that:

 \begin{equation}
   \lim_{s \to 0}\frac{\pot(t, \R)  - \pot(t+s^2,R)}{s^2} =  -\frac{\partial}{\partial t} \pot(t,\R)
   \label{eqn:dt}
   \end{equation}
 \begin{equation}
   \lim_{s \to 0}
   \frac{\pot(t+s^2,\R+s) - 2\pot(t+s^2,R)+\pot(t+s^2,\R-s)}{2s^2}
   =  \frac{1}{2} \frac{\partial^2}{\partial \R^2} \pot(t,\R)
\label{eqn:ddR2}
 \end{equation}

Combining Equations~\ref{eqn:difference},\ref{eqn:dt} and \ref{eqn:ddR2} we arrive at the Kolmogorov Backwards Equation:
\begin{equation} \label{eqn:Kolmogorov}
  \frac{\partial}{\partial t} \pot(t,\R)
  + \frac{1}{2} \frac{\partial^2}{\partial \R^2} \pot(t,\R)=0
\end{equation}

Repeating the same steps for the upper potential
\begin{equation}
   \frac{\pot(t, \R)  - \pot(t+s^2,R)}{s^2} =  \frac{\pot(t+s^2,\R+s(1+s)) - 2\pot(t+s^2,R)+\pot(t+s^2,\R-s(1+s))}{2s^2} \label{eqn:difference}
 \end{equation}
One finds that the limit is Eqn~\ref{eqn:Kolmogorov} as well.


The adversarial strategy corressponds to the distribution of sequences generated by a random walk.
The limit of random walk under the scaling $\Delta t = (\Delta R)^2$
the well known Brownian or Wiener process
(see~\cite{kac1947random}).

The min-max potential function $\phi(t,R)$ is a solution of the
differential Equation~(\ref{eqn:Kolmogorov}) with the boundary
condition $\finalpot =\pot(T,\cdot) \in \SP{4}$.

Alternatively, we can compute the potential for any
$0 \leq t \leq T$ and any $R$ using the equation
\begin{equation} \label{eqn:convol-with-normal}
\pot(t,\R)={\cal N}(\R,T-t) \odot \finalpot
\end{equation}

\section{Need to move back to $s_i$}
because learner's response to finite $s^i$ is not the limit.

It is not hard to verify that, in the limit $s_i \to 0$ the learner uses the distribution
$$P(t,R') = \left. \frac{\partial}{\partial R} \pot(t,R)  \right|_{R=R'}$$
This guarantees that the final average score is at
most $\pot(0,0)$.

A major limition of this result is that the horizon $\realT$ is set in
advance.  It is desirable that the potential is defined without
knowledge of the horizon.  In what follows we show that Hedge and
NormalHedge can both be used in such ``anytime'' algorithms.

Our solution is based on the observation that a potential function satisfies Eqn~(\ref{eqn:convol-with-normal}) if and only if it satisfies 
the Kolmogorov backwards PDE~(\ref{eqn:Kolmogorov}):
The potential $\finalPotR \in \SP{4}$ defines a boundary condition of the PDE.

We derive our anytime algorithm by finding solutions to the Kolmogorov
PDE that are not restricted in time, and that have a fixed parametric
form.  In other words, the evolution of the potential with time is
defined by changing the parameter values, without changing the form.

We describe two potential functions that are solutions of PDE. In the following section we use our general results to prove simultanous regret bounds

{\bf The exponential potential function} which corresponds to exponential
  weights algorithm corresponds to the following equation
\begin{equation} \label{eqn:exponential-potential}
    \pot_{\mbox{\tiny exp}}(\R,t) = e^{\sqrt{2} \eta \R - \eta^2 t}
\end{equation}

Equation~\ref{eqn:exponential-potential}.

For the standard (non simultanous) bound we fix $\epsilon$ and $t$,
choose $\eta = \sqrt{\frac{\ln (1/\epsilon)}{t}}$
and get a bound of the form 
  \begin{equation}
    \R_\epsilon \leq \sqrt{2 t \ln \frac{1}{\epsilon}}
  \end{equation}

  To derive a simultanous regret bound we fix the learning rate $\eta$ and take the reciprocal of the potential:
 \[
    G(R,t) \leq e^{\eta^2 t - \sqrt{2}\eta R}
 \]
Which holds for any $t$ and $R$. The bound $G(R,t)$ depends on $\eta$.

  
{\bf The NormalHedge potential function} parametrized by $\nu>0$ is:
\begin{equation} \label{eqn:NormalHedge}
  \pot_{\mbox{\tiny NH$(\nu)$}}(\R,t) = \begin{cases}
    \frac{1}{\sqrt{t+\nu}}\exp\left(\frac{\R^2}{2(t+\nu)}\right)
    & \mbox{if } \R \geq 0  \\
  \frac{1}{\sqrt{t+\nu}} & \mbox{if } \R <0
  \end{cases}
\end{equation}

The function $\pot_{\mbox{NH}}(\R,t)$ is not in $\SP{4}$, however, the
positive part $\R\geq 0$ is in $\SP{4}$ while the negative part
$\R \leq 0$ is a constant. A constant potential corresponds to zero weight which means that experts whose regret is negative are ignored by the learner. In this case the optimal adversarial is not unconstrained brownian motion, instead it is brownian motion with a reflective boundary at $R=0$.

  \subsection{A learner strategy with a variance-dependent bound}

  As shown in Lemma~\ref{lemma:adversary-prefers-extremes}, the
  adversary always prefers mixed strategies that assign zero
  probability for all steps other than $\pm 1$. Suppose, however, that
  the adversary is not worst-case optimal and chooses steps whose
  length is less than one. The following lemma gives a slightly
  different strategy for the learner, which guarantees a tighter bound
  for this case.

  \begin{lemma} \label{lemma:second-order-bound}
      The learner strategy:
      \begin{equation} \label{eqn:learner-strat-2}
      \learnerM^2(i-1,\R) =  \frac{1}{Z}
      \left. \frac{\partial}{\partial r} \right|_{r=\R} \pot(i,r)
      \end{equation}
      Where $Z$ is a normalization factor
      $$Z = \E{\R \sim \state(i)}{\left. \frac{\partial}{\partial r} \right|_{r=\R} \pot(i,r)}$$
      guarantees the following upper potential against any adversarial
      strategy $\adversM$
      \begin{equation} \label{eqn:value-iteration-upper}
        \upperpot(i-1, \R) = \pot(i,\R) + b(i,\R) \E{l \sim \adversM(i,\R)}{l^2}
      \end{equation}
      where $b(i,\R) = \pot(i,\R+1+c) -\pot(i,\R) - (1+c) \left. \frac{\partial}{\partial r} \right|_{r=\R} \pot(i,r)$
   \end{lemma}


The potentials and strategies defined above are scaled versions of the
integer time potential recursions defined in
Equations~(\ref{eqn:backward-iteration-lower},\ref{eqn:backward-iteration-upper-recursion})
and the strategies defined in Equations~(\ref{eqn:adv-strat-p},\ref{eqn:learner-strat-1}). Specifically, the games operate on lattices that we will now describe.

The adversarial strategy $\adversMb$ defines the following lattice over $i$ and $R$:
$$\Ilat{T}=\left\{ (i,2j-i) \left| 0 \leq i \leq T, 0 \leq j \leq i\right. \right\}$$

The $k$'th adversarial strategy $\adversMdk$ uses step size $s_k=\sqrt{\realT} 2^{-k}$ and time
increments $s_k^2=\realT 2^{-2k}$. We define the {\em game lattice}
for $k$ as the set of $(t,R)$ pairs that are reached by $\adversMdk$.
\begin{equation}  \label{eqn:game-lattice}
  \Klat{{\realT,k}}=\left\{ (t,\R) \left| t=i s_k^2, 0 \leq i \leq 2^{2k}, \R=(2j-i)s_k, 0 \leq j \leq i\right. \right\}
  \end{equation}
$\Ilat{T}$ is a special case of $\Klat{\realT,k}$ because setting
$\realT=T=2^{2k}$ we get that $s_k=s_k^2=1$ and  $\Klat{\realT,k} = \Ilat{T}$.

It is not hard to show that the lattices get finer with $k$, i.e. if  $j \leq k$,  $\Klat{\realT,j} \subseteq \Klat{\realT,k}$.


The following Lemma parallels Lemma~\ref{lemma:first-order-bound} for the integer time game.
\begin{lemma} \label{lemma:discrete-step-bound}
~\\
Let $i$ be an integer between $1$ and $T$

If $\lowerpotMdk(t_i,\R) \in \SP{2}$
\begin{enumerate}
\item  $\lowerpotMdk(t_{i-1},\R) \in \SP{2}$
\item The adversarial strategy~(\ref{eqn:adv-strat-dk})
  guarantees the recursion given in Eq.~(\ref{eqn:backward-iteration-lower-discrete})
\end{enumerate}

If $\upperpotMdk(t_i,\R) \in \SP{2}$
\begin{enumerate}
\item $\upperpotMdk(t_{i-1},\R) \in \SP{2}$
\item The learner strategy~(\ref{eqn:learner-strat-1c})
  guarantees the recursion given in Eq.~(\ref{eqn:backward-iteration-upper-recursion-discrete})
\end{enumerate}

\end{lemma}
\begin{proof}
The statement of the Lemma and the proof are scaled versions of
Lemma~\ref{lemma:first-order-bound} and its proof. The iteration step
is $s_k^2$ instead of $1$ while the loss/gain of an expert in a single
step is $[-s_k,s_k]$ instead of $[-1,+1]$.

One change worth noting is at the step from
Equation~(\ref{eqn:Pot-Update}) and Equation~(\ref{eqn:pot-upper}),
where the bound  $y-\ell(i) \in [-2,2]$ is replace by  $y-\ell(i) \in
[-s_k-s_k^2,s_k+s_k^2]$. This follows from the bound $|\ell(i)| \leq
s_k^2$ which is discussed in Section~\ref{sec:discrete}.
\end{proof}

\begin{theorem} \label{thm:DescreteGameExactValues}
  Let $\finalPotR \in \SP{2}$ be the final potential in the discrete
  time game. Fix $k$ and the step size $s_k=\sqrt{\realT} 2^{-k}$, and let $t_i=i s_k^2$ for $i=0,1,\ldots,2^{2k}$ and
  let $\R_0$ be a real value, then 
  \begin{itemize}
  \item
    The lower potential guaranteed by $\adversMdk$ is
    \begin{equation} \label{eqn:lower-potential-exact}
      \lowerpotMdk(t_i,\R_0) = \E{\R \sim \R_0 \oplus
        \Binom\paren{2^{2k}-i,s_k}}{\finalPotR(\R)}
      \end{equation}
  \item
    The upper potential guaranteed by $\learnerMdk$ is
    \begin{equation} \label{eqn:upper-potential-exact}
    \upperpotMdk(t_i,\R_0) =  \E{\R \sim \R_0 \oplus
      \Binom\paren{2^{2k}-i,s_k(1+s_k)}}{\finalPotR(\R)}
    \end{equation}
  \end{itemize}
\end{theorem}

Using Theorem~\ref{thm:DescreteGameExactValues} we can show that, as
$k \to \infty$, the upper lower potential converge to the same limit.

\begin{theorem} \label{thm:seq-of-adv-strategies}
  ~\\
  Fix $\realT$ and assume $\finalPotR \in \SP{2}$. Consider the sequence of upper
  and lower potentials  $\upperpotMdk,\lowerpotMdk$ for
  $k=0,1,2,\ldots$.

  Then for any  $0 < t \leq \realT$ and any $\R_0$:
  \begin{equation} \label{eqn:k-limit}
    \lim_{k \to \infty} \upperpotMdk(t,\R_0) =
    \lim_{k \to \infty}\lowerpotMdk(t,\R_0)=
    {\cal N}(\R_0,\realT-t) \odot \finalPotR
  \end{equation}
\end{theorem}

\begin{proof}

We first assume that $(t,\R_0) \in \Klat{j,\realT}$ and that $k\geq j$. We later expand to any $0 < t \leq \realT$ and any $\R_0 \in \reals$.
Consider Equation~\ref{eqn:upper-potential-exact} for $\learnerMdk$ and $\learnerMdj$,
keeping $t$ and $j$ constant and letting $k \to \infty$.

\begin{equation} \label{eqn:upper-potential-exact-j}
  \upperpotMdj(t,\R_0) =  \E{\R \sim \R_0 \oplus
    \Binom\paren{2^{2j}-i_j,s_j(1+s_j)}}{\finalPotR(\R)}
\end{equation}

\begin{equation} \label{eqn:upper-potential-exact-k}
  \upperpotMdk(t,\R_0) =  \E{\R \sim \R_0 \oplus
    \Binom\paren{2^{2k}-i_k,s_k(1+s_k)}}{\finalPotR(\R)}
\end{equation}
We rewrite the binomial factor in
Eq~(\ref{eqn:upper-potential-exact-k})
$$
 \Binom\paren{2^{2k}-i_k,s_k(1+s_k)} =
 \Binom\paren{2^{2(k-j)}\paren{2^{2j}-i_j},2^{j-k} s_j(1+2^{j-k} s_j)}
 $$
As $j$ is constant, $s_j$ is constant and so is $a_j \doteq
2^{2j}-i_j$. Multiplying the number of steps by the variance per step
we get
$$Var(\Binom_k) = 2^{2(k-j)}a_j\paren{2^{j-k} s_j(1+(2^{j-k} s_j)}^2
= a_j s_j^2 (1+(2^{j-k} s_j))^2 
$$

As $s_j,a_j$ are constants we get that $\lim_{k \to \infty}
Var(\Binom_k) = a_j s_j$.  From the central limit theorem we get that
for any $(t,\R_0) \in \Klat{j,\realT}$
\begin{equation} \label{eqn:conergence-for-k}
  \lim_{k \to \infty} \upperpotMdk(t,\R_0) \odot \finalPotR = {\cal
    N}(\R_0,\realT-t) \odot \finalPotR
\end{equation}

Our argument hold for all $(t,\R_0) \in \bigcup_{k=0}^\infty \Klat{k,\realT}$ which is dense in the set $0<t\leq \realT, \R_0 \in \reals$.
On the other hand, $\upperpotMdk(t,\R) \odot \finalPotR$ is continuous in both
$t$ and $R$, therefor Equation~(\ref{eqn:conergence-for-k})  holds for all $t$ and $\R$.

As similar (slightly simpler) argsument holds for the lower potential limit
$\lim_{k \to \infty} \lowerpotMdk(t,\R_0)$

\end{proof}

We have shown that in the limit $s \to 0$ the learner and the
adversary converge to the same potential function. In the next section
we show that this limit is the min/max solution by describing conditions
under which the adversary prefers using ever smaller steps size.


\section{Lower bound} \label{sec:lowerbound}

We now prove the lower bound given in Theorem~\ref{thm:LowerBound}.

The lower bound is based on a symmetric random walk strategy similar
to the one employed in Section~\ref{sec:disc-game-strategies}
Equation~\ref{eqn:adv-strat-dk}, with two main differences
\begin{enumerate}
\item we allow different step size on different iterations. We use a non-reflecctive random walk.
\end{enumerate}

As the strategy is a symmetric random walk the loss of each expert is
zero and the variance and the learner has no impact on the evolution of the game. For
$V^i \to \infty$ the distribution of the regrets is normal with mean 0
and variance $V_i$.  To prove this we appeal to Lindberg's
generalization of the central limit theorem.~(\cite{shiryaev1989probability}, page 328).
  
  \begin{theorem}[Lindberg CLT]
  Let $X_i$ be a squence of independent, zero mean, random variables with variance $\sigma_1^2,\sigma_2^2,\ldots$. Denote by $V_n  = \sum_{i=1}^n \sigma_i^2$

if for any $\epsilon>0$:
  \[
    \lim_{n \to \infty} \frac{1}{V_n} \sum_{i=1}^n \E{}{X_i^2 \;\;{\bf 1}_{X_i: |X_i| \geq \epsilon \sqrt{V_n}}}=0
  \]

 Then $\frac{\sum_{i=1}^n X_i}{\sqrt{V_n}}$ converges in probability to the standard normal ${\cal N}(0,1)$.
\end{theorem}

\noindent We now prove Theorem~\ref{thm:LowerBound}.
\begin{proof}
Suppose the adversary uses the symmetric random walk with unequal step sizes:
\begin{equation}
\adversMb(i,\R) =
  \begin{cases}
    +s_i & \mbox{ w.p. } \frac{1}{2}\\
    -s_i & \mbox{ w.p. } \frac{1}{2}\\
  \end{cases}
\end{equation}
for $s_i = \sqrt{\nu^i}$. The progression of the game for this
adversary is independent of the learning algorithm and of the
weighting $\{\cal I\}$. In particular
$V^T({\cal I}) = \sum_{i=1}^T \nu^i$. To simplify notation we replace
$V^T({\cal I})$ with $V^T$

Let $X_i$ denote the random variable which takes on the values
$\pm s_i$ with equal probabilities. Consider the distribution of $\frac{\sum_{i=1}^T X_i}{V^T}$
as $T \to \infty$.


Fix some $\epsilon>0$. As $s_i \leq 1$, $V^T \to \infty$ and
$|X_i| \leq 1$ there exists $T_{\epsilon}$ such that for any
$T \geq T_{\epsilon}$, and any $1 \leq i \leq T$,
$|X_i| < \epsilon \sqrt{V^T}$. Therefore Lindberg's expression
converges to zero as $T \to \infty$ for any fixed $\epsilon>0$
\[
  \frac{1}{V^T}\sum_{i=1}^T \E{}{X_i^2 \;\;{\bf 1}_{X_i: |X_i| \geq \epsilon \sqrt{V^T} } } = 
    \frac{1}{V^T}\sum_{i=1}^{T_\epsilon} \E{}{X_i^2 \;\;{\bf 1}_{X_i: |X_i| \geq \epsilon \sqrt{V^T} } }
    \leq \frac{\sum_{i=1}^{T_\epsilon} s_i^2}{\sum_{i=1}^{T} s_i^2}
    \stackrel{T \to \infty}{\longrightarrow} 0
  \]
  Lindberg's CLT implies that if we fix $a>0$ and set $R^T=a \sqrt{V^T}$ then the probability that
  $R>R_T$ is equivalent to the probability that ${\cal N}(0,1)$ assigns to the
  ray $[a,\infty]$.

  \[
    \int_a^\infty \frac{1}{\sqrt{2 \pi}} e^{-\frac{x^2}{2}} dx
    \geq \frac{1}{\sqrt{2 \pi}} e^{-\frac{a^2}{2}}
    \left( \frac{1}{a} - \frac{1}{a^3}\right)
  \]

Plugging in $a=\frac{R^T}{\sqrt{V^T}}$ we get the statement of theorem~\ref{thm:LowerBound}.
  
  \end{proof}

 

\bibliographystyle{plain}
\bibliography{ref.bib,bib.bib}

\appendix

\section{Proof of Theorem~\ref{thm:IntegerGameBounds}}


The following lemma states that these strategies guarantee the
corresponding potentials.
\begin{lemma} \label{lemma:first-order-bound}
~\\
Let $i$ be an integer between $1$ and $T$

If $\lowerpotb(i,\R) \in \SP{2}$
\begin{enumerate}
\item {\bf Positivity:} $\lowerpotb(i-1,\R) \in \SP{2}$
\item {\bf Adversary:} The adversarial strategy~(\ref{eqn:adv-strat-p})
  guarantees the recursion given in Eq.~(\ref{eqn:backward-iteration-lower})
\end{enumerate}

If $\upperpotb(i,\R) \in \SP{2}$
\begin{enumerate}
\item {\bf Positivity:} $\upperpotb(i-1,\R) \in \SP{2}$
\item {\bf Learner:} The learner strategy~(\ref{eqn:learner-strat-1})
  guarantees the recursion given in Eq.~(\ref{eqn:backward-iteration-upper-recursion})
\end{enumerate}

\end{lemma}

\begin{proof} {\bf Of Lemma~\ref{lemma:first-order-bound}}

    We prove each claim in turn
\begin{enumerate}
\item {\bf Positivity:} Follows from Lemma~\ref{lemma:SP-pos-comb}.
\item{\bf Adversary:} By symmetry adversarial strategy~(\ref{eqn:adv-strat-p}) guarantees that
  the aggregate loss~(\ref{eqn:aggregate-loss}) is zero regardless of
  the choice of the learner: $\ell(i)=0$.
  Therefor the state update~(\ref{eqn:state-update}) is equivalent to
  the symmetric random walk:
  $$\state(i) = \frac{1}{2} \paren{(\state(i) \oplus 1) + (\state(i)
    \ominus 1)}$$
  Which in turn implies that if the adversary plays $\adversM^*$
  and the learner plays an arbitrary strategy $\learnerM$
  \begin{equation} \label{eqn:lower}
    \lowerpotb(i-1,\R) = \frac{\lowerpotb(i,\R-1)+\lowerpotb(i,\R+1)}{2}
  \end{equation}
  As this adversarial strategy is oblivious to the learner's strategy, it
  guarantees that the average value at iteration $i$ is {\em equal} to the
  average of the lower value at iteration $i$.
\item {\bf Learner:}
  Plugging learner's strategy~(\ref{eqn:learner-strat-1})
  into equation~(\ref{eqn:aggregate-loss}) we find that
 \begin{equation} \label{eqn:ell-optimal-learner}
   \ell(i) = \frac{1}{Z_{i}} \E{\R \sim \state(i)}{\paren{\upperpotb(i,\R+2)-\upperpotb(i,\R-2)}
   \Bias(i,\R)}
\end{equation}
  Consider the score at iteration $i$ when the learner's strategy
  is $\learnerM^*$ and the adversarial strategy  $\adversM$ is arbitrary
     \begin{equation} \label{eqn:Pot-Update}
    \score_{\learnerM^*,\adversM}(i,\R) = \E{\R \sim \state(i)}{ \E{y \sim
      \adversM(i)(\R)}{\pot(i,\R+y-\ell(i))}}
  \end{equation}
  As $\pot(i,\cdot)$ is convex and as $y-\ell(i) \in [-2,2]$,
  \begin{equation} \label{eqn:pot-upper}
    \upperpotb(i-1,\R+y) \leq \frac{\upperpotb(i,\R+2)+\upperpotb(i,\R-2)}{2} +
    (y-\ell(i)) \frac{\upperpotb(i,\R+2)-\upperpotb(i,\R-2)}{2}
    \end{equation}
  Combining the equations~(\ref{eqn:ell-optimal-learner}) and~(\ref{eqn:Pot-Update}) we find that
  \begin{eqnarray}
  \score_{\learnerM^*,\adversM}(i,\R)&=&\E{\R \sim \state(i)}{\E{y \sim \adversM(i)(\R)}{\upperpotb(i,\R+y-\ell(i))}}\\
  &\leq & \E{\R \sim \state(i)}{\frac{\upperpotb(i,\R+2)+\upperpotb(i,\R-2)}{2}}\\
  &+&
  \E{\R \sim \state(i)}{\E{y \sim \adversM(i)(\R)}{(y-\ell(i)) \frac{\upperpotb(i,\R+2)-\upperpotb(i,\R-2)}{2}}} \label{eqn:zero-term}
  \end{eqnarray}
  
The final step is to show that the term~(\ref{eqn:zero-term}) is equal
to zero. As $\ell(i)$ is a constant with respect to $\R$ and $y$ the
term~(\ref{eqn:zero-term}) can be written as:
\begin{eqnarray}
&&\E{\R \sim \state(i)}{\E{y \sim \adversM(i)(\R)}{(y-\ell(i))
   \frac{\upperpotb(i,\R+2)-\upperpotb(i,\R-2)}{2}}}\\
&=&
\E{\R \sim \state(i)}{\Bias(i,\R)
    \frac{\upperpotb(i,\R+2)-\upperpotb(i,\R-2)}{2}}\\
  &-& \ell(i) \E{\R \sim \state(i)}{
    \frac{\upperpotb(i,\R+2)-\upperpotb(i,\R-2)}{2}}\\
  &=& 0
\end{eqnarray}
\end{enumerate}
\end{proof}

Repeating the induction steps of Lemma~\ref{lemma:first-order-bound}
from $i=T$ to $i=1$ yields the proof of the theorem.


\section{Divided differences of a function} \label{sec:divdiff}

\begin{lemma} \label{lemma:differences}
Let $f$ be a function in $\SP{4}$, let $x$ be any real and $s$ be a positive real. Define
\begin{equation}
  h(x,s) =  \frac{1}{24s^4} \paren{f(x)- 4f(x+s)+ 6f(x+2s)-
    4f(x+3s)+ f(x+4s)}
\end{equation}
Then $h(x,s)>0$.
\end{lemma}

The function $h(x,s)$ has a special form called ``divided difference''
that has been extensively studied ~\cite{popoviciu1965certaines,butt2016generalization, de2005divided}.
and is closely related to to derivatives of different orders. properties of divided difference are at the core of the proof of lemma~\ref{lemma:differences}

\iffalse
We conclude that if $\pot(t',\R)$ has a strictly positive fourth
derivative then $\pot_{k+1}(t,\R) > \pot_{k}(t,\R)$ for all $\R$, proving
the first part of the lemma.

The second part of the lemma follows from the fact that
both $\pot_{k+1}(t,\R)$ and $\pot_{k}(t,\R)$ are convex combinations of
$\pot(t,\R)$ and therefor retain their continuity and convexity properties.

A function $\finalPot{}$ that satisfies
inequality~\ref{eqn:4thOrderConvex} is said to be {\em 4'th order convex}
(see details in in~\cite{butt2016generalization}).
\fi

Following\cite{butt2016generalization} we give a brief review of
divided differences and of $n$-convexity.

Let $f:[a,b] \to \reals$ be a function from the segment $[a,b]$ to the
reals.

\begin{definition}[$n$'th order divided difference of a function] \label{def:recurse-divided-difference}
  The $n$'th order divided different of a function $f:[a,b] \to
  \reals$ at mutually distinct and ordered points $a \leq x_0 < x_1
  < \cdots < x_n \leq b$
  defined recursively by
  \[ [x_i; f] = f(x_i), \; i \in 0,\ldots n,\]
  \[ [x_0,\ldots,x_n;f] =
    \frac{[x_1,\ldots,x_n;f]-[x_0,\ldots,x_{n-1};f]}{x_n-x_0} \]
\end{definition}

\begin{definition}[$n$-convexity]
 A function $f:[a,b] \to \reals$ is said to be $n$-convex  $n \geq 0$
 if and only if for all choices of $n+1$ distinct points: $a \leq x_0 < x_1
  < \cdots < x_n \leq b$, $[x_0,\ldots,x_n;f]\geq 0$ holds.
\end{definition}
$n$-convexity is has a close connection to the sign of $f^{(n)}$ - the $n$'th
derivative of $f$, this connection was proved in 1965 by
popoviciu~\cite{popoviciu1965certaines}.
\begin{theorem} \label{thm:popo}
If $f^{(n)}$ exists then f is $n$-convex if and only if $f^{(n)}\geq 0$.
\end{theorem}

The next lemma states that the function $g(\R)>0$ as defined in
Equation~(\ref{eqn:divdiff}).

\begin{proof} {\bf of Lemma~(\ref{lemma:differences})}

All we need to do is show that the function $h$ is a fourth order divided difference.
define $(x_0,x_1,x_2,x_3,x_4) = (x,x+a,x+2a,x+3a,x+4a)$ and apply Definition~\ref{def:recurse-divided-difference} five times.

\begin{enumerate}
\item
$$[x_i;f] = f(x_i)$$
\item
  $$[x_i,x_{i+1};f]=\frac{f(x_{i+1})-f(x_i)}{s}$$
\item
  $$[x_i,x_{i+1},x_{i+2};f] =
  \frac{\frac{f(x_{i+2})-f(x_{i+1})}{s}-\frac{f(x_{i+1})-f(x_i)}{s}}{2s}
  =\frac{f(x_{i+2})-2f(x_{i+1})+f(x_i)}{2 s^2}
  $$
\item
  \begin{eqnarray*}
    [x_i,x_{i+1},x_{i+2},x_{i+3};f]& = &
    \frac{\frac{f(x_{i+3})-2f(x_{i+2})+f(x_{i+1})}{2
    s^2}-\frac{f(x_{i+2})-2f(x_{i+1})+f(x_i)}{2
    s^2}}{3s}\\
    &=& \frac{f(x_{i+3}) -3f(x_{i+2})+3f(x_{i+1})-f(x_i)   }{6 s^3}
  \end{eqnarray*}
\item
  \begin{eqnarray*}
    [x_i,x_{i+1},x_{i+2},x_{i+3},x_{i+4};f]& = &
    \frac{\frac{f(x_{i+4}) -3f(x_{i+3})+3f(x_{i+2})-f(x_{i+1})   }{6 s^3}
    - \frac{f(x_{i+3}) -3f(x_{i+2})+3f(x_{i+1})-f(x_i)   }{6 s^3}}
    {4s}\\
    &=& \frac{f(x_{i+4})-4f(x_{i+3})+6f(x_{i+2})-4f(x_{i+1})+f(x_i)}{24s^4}
  \end{eqnarray*}
\end{enumerate}
We have shown that $h=f^{(4)}$. The statement of the Lemma follows from the fact that $f \in SP{4}$ together with Theorem~\ref{thm:popo}.
\end{proof}
  

\section{Proof of Theorem~\ref{thm:variancebound}}

We start with two technical lemmas
\begin{lemma} \label{lemma:infiniteexpectations}
Let $f(x) \in \SP{2}$, i.e. $f(x), f'(x),f''(x) >0$ for all $x \in
\reals$, let $h(x)$ be a uniformly bounded function: $\forall x,\;\; |h(x)|<1$.
Let $\state$ be a distribution over $\reals$.
If $\E{x \sim \state}{f(x)}$ is well-defined (and finite) , then 
$\E{x \sim \state}{h(x) f'(x)}$ is well defined (and finite) as well.
\end{lemma}
\begin{proof}
Assume by contradiction that $\E{x \sim \state}{h(x) f'(x)}$ is
undefined. Define $h^+(x) = \max(0,h(x))$.
As $f'(x)>0$, this implies that either $\E{x \sim \state}{h^+(x)
  f'(x)}=\infty$ or $\E{x \sim \state}{(-h)^+(x) f'(x)}=\infty$ (or both). 

Assue wlog that $\E{x \sim \state}{h^+(x) f'(x)}=\infty$. As
$f'(x)>0$ and $0 \leq h^+(x) \leq 1$ we get that $\E{x \sim
  \state}{f'(x)}=\infty$.
As $f(x+1) \geq f'(x)$ we get that $\E{x \sim
  \state}{f(x)}=\infty$ which is a contradiction.
\end{proof}

\iffalse
$\epsilon>0$ such that
for any $v>0$ there exists $|h_v(x)| < v$ such that 
$|\E{x \sim \state}{h_v(x) f'(x)}|>\epsilon$. Define $h'_v =
\frac{h_v}{v}$. Clearly $|h'_v(x)|<1$ and we have that
$$|\E{x \sim \state}{h'_v(x) f'(x)}|>\frac{\epsilon}{v}$$

Define the positive parts of $h_v$ to be $h_v^+(x) = \max(0,h'_v(x))$
and $h_v^-(x) = \max(0,-h'_v(x))$. As $f'(x)>0$ we that either
\[
  \E{x \sim \state}{h^+_v(x) f'(x)}>\frac{\epsilon}{v}
\]
or 
\[
  \E{x \sim \state}{h^-_v(x) f'(x)}>\frac{\epsilon}{v}
\]
Wlog assume the first. As $f \in \SP{2}$, we have that $f(x+1) \geq f'(x)$. Therefore
 $\E{x \sim \state}{f(x+1)|h_v^(x)>0)}=\infty$. As $f(x)>0$ for all $x$ we
 find that $\E{x \sim \state}{f(x)}=\infty$ which contradicts the
 Lemma's assumption.
\end{proof}
\fi

\newcommand{\Dx}{\Delta x}
\newcommand{\Dy}{\Delta y}
\begin{lemma} \label{lemma:Taylor2D}
Let $f(x,y)$ be a differentiable function with continuous derivatives
up to degree three. Then
\begin{eqnarray}
  &&f(x_0+\Dx,y_0+\Dy) = f(x_0,y_0)
  + \atI{\frac{\partial}{\partial x}} \Dx 
  + \atI{\frac{\partial}{\partial y}} \Dy \\
  &+&\frac{1}{2} \atI{\frac{\partial^2}{\partial x^2}} \Dx^2
      +\atI{\frac{\partial^2}{\partial x\partial y}} \Dx\Dy
      +\frac{1}{2} \atI{\frac{\partial^2}{\partial y^2}} \Dy^2\\
  &+&\frac{1}{6} \atII{\frac{\partial^3}{\partial x^3}} \Dx^3
      +\frac{1}{2} \atII{\frac{\partial^3}{\partial x^2 \partial y}} \Dx^2\Dy\\
  &&+ \frac{1}{2} \atII{\frac{\partial^3}{\partial x \partial y^2}} \Dx\Dy^2
    + \frac{1}{6} \atII{\frac{\partial^3}{\partial y^3}} \Dy^3
\end{eqnarray}
for some $0\leq t \leq 1$.
\end{lemma}
\begin{proof} {\em of Lemma~\ref{lemma:Taylor2D}} 
Let $F:[0,1] \to \reals$ be defined as  $F(t)=f(x(t),y(t))$ where
$x(t) = x_0+t\Dx$ and $y(t)=y_0+t\Dy$. Then $F(0)=f(x_0,y_0)$ and
$F(1)=f(x_0+\Dx,y_0+\Dy)$. It is easy to verify that
$$ \frac{d}{dt}F(t)
=\frac{\partial}{\partial x} f(x(t),y(t))\Dx
+ \frac{\partial}{\partial y} f(x(t),y(t))\Dy
$$
and that in general:
\begin{equation} \label{eqn:d.dn.F}
\frac{d^n}{d t^n} F(t) = \sum_{m=1}^n {n \choose m}
\frac{\partial^n}{\partial x^m \partial y^{n-m}} f(x_0+t \Dx,y_0+t\Dy)
\Dx^m \Dy^{n-m}
\end{equation}
As $f$ has partial derivatives up to degree 3, so does $F$. Using the
Taylor expansion of $F$ and the intermediate point theorem we get that
\begin{equation} \label{eqn:Taylor.F}
  f(x_0+\Dx,y_0+\Dy) = F(1) = F(0)+\frac{d}{dt}F(0)
  +\frac{1}{2}\frac{d^2}{dt^2}F(0)
  +\frac{1}{6}\frac{d^3}{dt^3}F(t')
\end{equation}
Where $0 \leq t' \leq 1$. Using Eqn~(\ref{eqn:d.dn.F}) to expand each
term in Eqn.~(\ref{eqn:Taylor.F}) completes the proof.
\end{proof}


\begin{proof} {\em of Theorem~\ref{thm:variancebound}}\\
We prove the claim by an upper bound on the increase of potential that holds for any iteration $1 \leq i \leq n$:
\begin{equation} \label{proof:onestep}
\score(\state(t_{i+1})) \leq \score(\state(t_i)) + a s_i^3 \mbox{ for some constant } a>0
\end{equation}
Summing inequality~(\ref{proof:onestep}) over all iterations we get that 
\begin{equation} \label{proof:allsteps}
\score(\state(T)) \leq \score(\state(0)) + c \sum_{i=1}^n s_i^3 \leq 
\score(\state(0)) + a s \sum_{i=1}^n s_i^2 = 
\score(\state(0)) + a s T
\end{equation}
From which the statement of the theorem follows.

We now prove inequality~(\ref{proof:onestep}). 
We use the notation $r=y -\ell(i)$ to denote the instantaneous regret at iteration $i$. 


Applying Lemma~\ref{lemma:Taylor2D} to
$\pot(t_{i+1},\R_{i+1})=\pot(t_i+\deltat_i,\R_i+r_i)$  we get
\begin{eqnarray} 
    \pot(t_i+\deltat_i,\R_i+r_i) & =&  
    \pot(t_i,\R_i)\\
    &+&\at{\frac{\partial}{\partial \rho}} r_i \\
    &+&\at{\frac{\partial}{\partial \tau}}  \deltat_i \\
    &+& \frac{1}{2} \at{\frac{\partial^2}{\partial \rho^2}} r_i^2 \\
    &+& \at{\frac{\partial^2}{\partial r \partial \tau}} r_i \deltat_i \label{term:Taylor_rdt}\\
    &+& \frac{1}{2} \at{\frac{\partial^2}{\partial \tau^2}} \deltat_i^2 \label{term:Taylor_dtsquare}\\
    &+& \frac{1}{6} \att{\frac{\partial^3}{\partial \rho^3}} r_i^3 \label{term:Taylor_r3}\\
    &+& \frac{1}{2} \att{\frac{\partial^3}{\partial \rho^2 \partial \tau}} r_i^2\deltat_i \label{term:Taylor_r2t}\\
    &+& \frac{1}{2} \att{\frac{\partial^3}{\partial \rho \partial \tau^2}} r_i\deltat_i^2 \label{term:Taylor_rt2}\\
    &+& \frac{1}{6} \att{\frac{\partial^3}{\partial \tau^3}} \deltat_i^3 \label{term:Taylor_t3}
\end{eqnarray}
for some $0 \leq g \leq 1$.

By assumption $\pot$ satisfies the Kolmogorov backward equation:
\begin{equation*} 
  \frac{\partial}{\partial \tau} \pot(\tau,\rho)
  = -\frac{1}{2} \frac{\partial^2}{\partial r^2} \pot(\tau,\rho)
\end{equation*}
Combining this equation with the exchangability of the order of
partial derivative (Clairiaut's Theorem) we can substitute all
partial derivatives with respect to $\tau$ with partial derivatives
with respect to $\rho$ using the following equation.
\[
  \frac{\partial^{n+m}}{\partial \rho^n \partial \tau^m} \pot(\tau,\rho)=
  (-1)^m \frac{\partial^{n+2m}}{\partial \rho^{n+2m}} \pot(\tau,\rho)
\]
Which yields
\begin{eqnarray}
      \pot(t_i+\deltat_i,\R_i+r_i) & =&  
    \pot(t_i,\R_i)\\
    &+& \at{\frac{\partial}{\partial \rho}} r_i \label{term:coll1}\\
    &+& \at{\frac{\partial^2}{\partial \rho^2}} \paren{\frac{r_i^2}{2}-\deltat_i} \label{term:coll2}\\
    &-& \at{\frac{\partial^3}{\partial \rho^3}} r_i \deltat_i \label{term:coll3}\\
    &+& \frac{1}{2} \at{\frac{\partial^4}{\partial \rho^4}} \deltat_i^2 \label{term:coll4}\\
    &+& \frac{1}{6} \att{\frac{\partial^3}{\partial \rho^3}} r_i^3 \label{term:coll5}\\
    &-& \frac{1}{2} \att{\frac{\partial^4}{\partial \rho^4}} r_i^2\deltat_i \label{term:coll6}\\
    &+& \frac{1}{2} \att{\frac{\partial^5}{\partial \rho^5}} r_i\deltat_i^2 \label{term:coll7}\\
    &-& \frac{1}{6} \att{\frac{\partial^6}{\partial \rho^6}} \deltat_i^3 \label{term:coll8}
\end{eqnarray}

  From the assumption that the game is $(n,s,T)$-bounded we get that 
  \begin{enumerate}
  \item $|r_i| \leq s_i +c s_i^2 \leq 2 s_i$
  \item $\deltat_i = s_i^2 \leq s^2$
    % \item $\sum_i \deltat_i=T$
  \end{enumerate}

  given these inequalities we can rewrite the second factor in each
  term as follows, where $|h_i(\cdot)|\leq 1$
  \begin{itemize}
  \item {\bf For~(\ref{term:coll1}):}
    $r_i=2s_i\frac{r_i}{2s_i}=2s_ih_1(r_i)$.
  \item {\bf For~(\ref{term:coll2}):}
    $r_i^2 - \frac{1}{2}\deltat_i = 4s_i^2\frac{r_i^2 -
      \frac{1}{2}\deltat_i}{4s_i^2} = 4s_i^2 h_2(r_i,\deltat_i)$
  \item {\bf For~(\ref{term:coll3}):} $r_i \deltat_i = 2s_i^3
    \frac{r_i \deltat_i}{2s_i^3} = 2s_i^3 h_3(r_i,\deltat_i)$
  \item {\bf For~(\ref{term:coll4}):} $\deltat_i^2 =
    s_i^4\frac{\deltat_i^2}{s_i^4} = s_i^3 h_4(\deltat_i)$
  \item {\bf For~(\ref{term:coll5}):} $r_i^3 = 8s_i^3
    \frac{r_i^3}{8s_i^3} = 8s_i^3 h_5(r_i,\deltat_i)$
  \item {\bf For~(\ref{term:coll6}):} $r_i^2 \deltat_i = 4s_i^4
    \frac{r_i^2 \deltat_i}{4s_i^4} = 4s_i^3 h_6(r_i,\deltat_i)$
  \item {\bf For~(\ref{term:coll7}):} $r_i \deltat_i^2 = 2s_i^5
    \frac{r_i \deltat_i^2}{2s_i^5}$
  \item {\bf For~(\ref{term:coll8}):} $\deltat_i^3 = s_i^6 \frac{\deltat_i^3}{s_i^6}$
\end{itemize}
  We therefor get the simplified equation
  
  \begin{eqnarray*} 
     \pot(t_i+\deltat_i,\R_i+r_i) & =&  \pot(t,\R)+\at{\frac{\partial}{\partial r}} r
    + \at{\frac{\partial}{\partial t}} \deltat \\
                                  &+& 
                                      \frac{1}{2}  \at{\frac{\partial^2}{\partial r^2}} r^2\\
                                  &+& \at{\frac{\partial^2}{\partial r \partial t}} r_i \deltat_i \label{term:Taylor_collected_rdt}\\
                                  &+& \frac{1}{6} \at{\frac{\partial^3}{\partial r^3}} r_i^3 \label{term:Taylor_collected_r3}
                                      + O(s^4)
\end{eqnarray*}

and therefor
  \begin{eqnarray} 
     \pot(t_i+\deltat_i,\R+r) &=& \pot(t_i,\R) +
                                  \at{\frac{\partial}{\partial r}} r
                                  \nonumber \\
    &+& \at{\frac{\partial^2}{\partial r^2}} (r^2 - \deltat_i) +
        O(s^3) \label{eqn:Taylor}
\end{eqnarray}

Our next step is to consider the expected value of~(\ref{eqn:Taylor}) wrt $\R \sim \state(t_i)$,
$y \sim \adversM(t_i,\R)$ for an arbitrary adversarial strategy
$\adversM$.

We will show that the expected potential does not increase:
\begin{equation} \label{eqn:deltatislargeenough}
     \E{\R \sim \state(t_i)}{ \E{y \sim \adversM(t_i,\R)}{\pot(t_i+\deltat_i,\R+y-\ell(t_i))}} \leq \E{\R \sim \state(t_i)}{\pot(t_i,\R)}
\end{equation}

Plugging Eqn~(\ref{eqn:Taylor}) into the LHS of
Eqn~(\ref{eqn:deltatislargeenough}) we get
\begin{eqnarray}
  \lefteqn{\E{\R \sim \state(t_i)}{\E{y \sim \adversM(t_i,\R)}{\pot(t_i+\deltat_i,\R+y-\ell(t_i))}}} \\
  &=& \E{\R \sim \state(t_i)}{\pot(t_i,\R)} \label{eqn:contin0}\\
  &+& \E{\R \sim \state(t_i)}{\E{y \sim \adversM(t_i,\R)}{\at{\frac{\partial}{\partial r}} (y-\ell(t_i))}} \label{eqn:contin1}\\
  &+& \E{\R \sim \state(t_i)}{\E{y \sim
      \adversM(t_i,\R)}{\at{\frac{\partial^2}{\partial r^2}}
      ((y-\ell(t_i))^2 - \deltat_i)}}
  \label{eqn:contin2}\\
  &+& O(s^3) \label{eqn:contin3}
\end{eqnarray}
Some care is needed here. we need to show that the expected value
are all finite. We assume that the expected potential
(Eqn~({eqn:contin0}) is finite. Using
Lemma~\ref{lemma:infiniteexpectations} this implies that the expected
value of higher derivatives of $\frac{\partial}{\partial \R} \pot(\R)$
are also finite.\footnote{I need to clean this up and find an argument
  that the expected value for mixed derivatives is also finite.}


To prove inequality~(\ref{proof:onestep}), we need to show that the
terms~\ref{eqn:contin1} and \ref{eqn:contin2} are smaller or equal to
zero.
~\\~\\~\\
{\bf Term~(\ref{eqn:contin1}) is equal to zero:}\\
As $\ell(t_i)$ is a constant
relative to $\R$ and $y$, and $\at{\frac{\partial}{\partial r}}$ is a
constant with respect to $y$ we can rewrite~(\ref{eqn:contin1}) as
\begin{equation} \label{eqnterm1.1}
  \E{\R \sim \state(t_i)}{\at{\frac{\partial}{\partial r}}
    \E{y \sim \adversM(t_i,\R)}{y} }
- \ell(t_i) \E{\R \sim \state(t_i)}{\at{\frac{\partial}{\partial r}}}
\end{equation}

Combining the definitions of $\ell(t)$~(\ref{eqn:ell-discrete}) and~
and the learner's strategy
$\learnerM^{cc}$~(\ref{eqn:learner-strat-cc}) we get that
\begin{eqnarray}
\ell(t_i) &=& \E{\R \sim \state{t_i}}{\frac{1}{Z}
              \at{\frac{\partial}{\partial r}}
              \E{y \sim \adversM(i,\R)}{y}} \mbox{ where }
              Z=\E{\R \sim \state{t_i}}{\frac{1}{Z}
              \at{\frac{\partial}{\partial r}}}
              \label{eqnterm1.2}
\end{eqnarray}

Plugging~(\ref{eqnterm1.2}) into (\ref{eqnterm1.1}) and recalling the
requirement that $\ell(t_i)<\infty$ we find that
term~(\ref{eqn:contin1}) is equal to zero.


~\\~\\~\\
{\bf Term~(\ref{eqn:contin2}) is equal to zero:}\\
As $\deltat_i$ is a constant relative to $y$, we can take it
outside the expectaation to give 
\begin{equation} \label{eqn:term2.1}
  \E{\R \sim \state(t_i)}{\E{y \sim
      \adversM(t_i,\R)}{\at{\frac{\partial^2}{\partial r^2}}
      (y-\ell(t_i))^2}}
  - \deltat_i  \;  \E{\R \sim \state(t_i)}{\at{\frac{\partial^2}{\partial r^2}}}
\end{equation}
On the other hand, from the choice of $\deltat_i$ given in
Eqn~(\ref{eqn:deltat}) we get
\begin{equation} \label{eqn:term2.2}
  \deltat_i \; \E{\R \sim \state(t_i)}{\at{\frac{\partial^2}{\partial r^2}}} = 
  \E{\R \sim \state(t_i)}{\E{y \sim \adversM(t_i,R)}{
      \at{\frac{\partial^2}{\partial r^2}} (y-\ell(t_i))^2}}
\end{equation}
Combining~(\label{eqn:term2.1}) and~(\ref{eqn:term2.2}) we
find that (\ref{eqn:contin2}) is zero.

Finally (\ref{eqn:contin3}) is negligible relative to the other terms
as $s \to 0$.
\end{proof} 


\end{document}

%% deleted stuff


The next Lemma is the main part of the proof of
Theorem~(\ref{thm:IntegerGameBounds}). We use the backward induction
from Theorem~(\ref{thm:backward-recursion}) To compute upper and lower
potentials (Equations~(\ref{eqn:upperPotentials},\ref{eqn:lowerPotentials})) for
Strategies~(\ref{eqn:adv-strat-p}) and~(\ref{eqn:learner-strat-1})

The last iteration of the game: $i=T$ is the first step of the
backward induction. The upper and lower bounds are both set equal to
the first step in the backward induction we define
$$  \lowerpotb(T,\R) = \upperpotb(T,\R) = \pot(T,\R) $$

\iffalse
\begin{lemma} \label{lemma:first-order-bound}
  If $\pot(i,\R) \in \SP{2}$
  \begin{enumerate}
    \item The adversarial strategy~(Eq~(\ref{eqn:adv-strat-p}))
    guarantees the lower potential
 \begin{equation} \label{eqn:backward-iteration-lower}
   \lowerpotb(i, \R) = \frac{\lowerpotb(i,\R+1) + \lowerpotb(i,\R-1)}{2}
 \end{equation}
   
    \item The learner strategy~(Eq~(\ref{eqn:learner-strat-1}))
      guarantees the upper potential 
      \begin{equation} \label{eqn:backward-iteration-upper-recursion}
        \upperpotb(i, \R) = \frac{\upperpotb(i,\R+2) + \upperpotb(i,\R-2)}{2}
      \end{equation}
    \end{enumerate}
\end{lemma}
\fi


  \begin{eqnarray}
  \upperpotd(t_i,\R)&=&\E{\R \sim \state(i)}{\E{y \sim \adversM(t_i,\R)}{\upperpotd(t_i,\R+y-\ell(t_i))}}\\
  &\leq & \E{\R \sim \state(t_i)}{\frac{\upperpotd(i,\R+s_k(1+s_k))+\upperpotd(i,\R-s_k(1+s_k))}{2}}\\
  &+&
      \E{\R \sim \state(i)}{\E{y \sim \adversM(i)(\R)}{(y-\ell(i))
      \frac{\upperpotd(i,\R+s_k+s_k^2)+\upperpotd(i,\R-s_k-s_k^2)}{2}}}
  \end{eqnarray}


(Eq~\ref{eqn:learner-strat-1}). We follow the same line of argument as the second part of the proof of
Lemma~\ref{lemma:first-order-bound} to give a recursion for the upper
potential. The critical difference between the integer game is and the
discrete game is that in the discrete game $\ell(t_i)\leq s_i^2$ which
implies that $(y-\ell(t_i)) \in [-s_k(1+s_k),s_k(1+s_k)]$. This yields 


  Following the same line of argument as the first part of the proof of
Lemma~\ref{lemma:first-order-bound} we consider the time points:
$t_i=i s_k^2=i 2^{-2k}\realT$ for $i=0,1,\ldots,2^{2k}$.

  \begin{equation} \label{eqn:lower-discrete}
    \lowerpotd(t_{i-1},\R) = \frac{\lowerpotd(t_i,\R-s_k)+\lowerpotd(t_i,\R+s_k)}{2}
  \end{equation}


  %%%%%

  
Follow the proof of
  Lemma~\ref{lemma:first-order-bound}.Derive upper and lower scores
  for the two strategies. Show that they converge to the same thing as
  $k \to \infty$.




We start with the high-level idea. Consider iteration $i$ of the
continuous time game. We know that the adversary prefers $s_i$ to be
as small as possible. On the other hand, the adversary has to choose
some $s_i>0$. This means that the adversary always plays
sub-optimally. Based on $s_i$ the learner makes a choice and the
adversary makes a choice. As a result the current state $\state(t_{i})$
is transformed to $\state(t_i)$. To choose it's strategy, the learner
needs to assign value possible states $\state(t_i)$. How can she do
that? By assuming that in the future the adversary will play
optimally, i.e. setting $s_i$ arbitrarily small. While the adversary
cannot be optimal, it can get arbitrarily close to optimal, which is
Brownian motion.

Note that the learner chooses a distribution {\em after} the adversary
set the value of $s_i$. The discrete time version of $\learnerM^1$

In the discrete time game the adversary has an additional choice, the
choice of $s_i$. Thus the adversary's strategy includes that choice.
There are two constraints on this choice: $s_i \geq 0$ and
$\sum_{i=1}^n s_i^2 = T$. Note that even that by setting $s_i$
arbitrarily small, the adversary can make the number of steps - $n$ -
arbitrarily large. We will therefor not identify a single adversarial
strategy but instead consider the supremum over an infinite sequence
of strategies.

%%% Local Variables:
%%% mode: latex
%%% TeX-master: t
%%% End:

\iffalse
  We study two particular potential functions which achieve good simultanous regret bounds.

\subsection{Min/Max solution}

Theorem~\label{thm:simulBoundAveragePot} shows a one-to-one relationship between a simultanous regret bound
$B(\R)$ and the average potential function $\phi(\R)=B(\R)^{-1}$. We therefor ask what is the smallest possible $B(\R)$. Recall that the brownian motion adversary forces a cumulative loss of zero on any learner. The SBR for Brownian motion is the tail of the normal distribution:
\newcommand{\erf}{\mbox{erf}}
\begin{equation}
  B^*(\R) = \frac{1}{\sqrt{2 \pi}}  \int_{\R}^\infty e^{-\frac{x^2}{2\realT}}dx
  = \frac{1}{2} \paren{1-\erf\paren{\frac{R}{\sqrt{2\realT}}}}
\end{equation}
The final potential function that correspond to
\begin{equation}
  \finalPotR^*(\R) = (B^*(\R))^{-1} = \frac{2}{1-\erf\paren{\frac{\R}{\sqrt{2\realT}}}}
\end{equation}
One can check that $\finalPotR^* \in \SP{4}$ and therefor Brownian motion is the optimal strategy and
$$\pot^*(t,\R)= \E{\rho \sim {\cal N}(\R,\realT-t)}{\frac{2}{1-\erf\paren{\frac{\rho}{\sqrt{2\realT}}}}}$$
The bound on the final potential is:
\[
  \pot^*(0,0) =  \E{\rho \sim {\cal N}(0,\realT)}{\frac{2}{1-\erf\paren{\frac{\rho}{\sqrt{2\realT}}}}}
\]

\fi

\subsection{Anytime Potentials} \label{sec:stable}



\iffalse
\begin{itemize}
\item We provide min-max analysis for potential and quantile based online
  algorithms with a finite known horizon and a potential function with
  four strictly positive derivativs. In particular, we show that
  Brownian motion is the optimal adversarial strategy for any such
  potential function.
\item We show how to remove the finite horizon assumption and derive
  both exponential weights and NormalHedge potentials.
\item We provide second order bounds on the parameter-free normal-Hedge algorithm
  that depend only on the percentile $\epsilon$ and a variant of the cumulative
  variance. We show that this bound is the optimal bound for any
  potential based algorithm.
\item We construct a lower bound that holds for any online algorithm
  and is very close to the normal hedge upper bound.
\end{itemize}

We start with the bounded horizon case. Fixing a potential function at
the end of the game $\pot(T,\R)$ and the strategies used by the
learner and the adversary, we define potential functions $\pot(i,\R)$
for iterations $i = T-1,T-2,\ldots,0$ such that the score $\score(t)$ is guaranteed to 
be equal for all of the iterations.
\[
  \score^T = \score(T-1)=\cdots=\score(0)
\]
This allows us to analyze the game one iteration at a time and
construct good strategies for both sides. We name this potential based
game the {\em Integer Time Game}, the analysis of this game is given
in Section~\ref{sec:int-time-game}. The analysis assumes only that the
final potential $\pot(T,\R)$ is strictly positive and has strictly positive
first and second derivatives (We denote the set
of functions that have $0,\ldots,k$ strictly positive derivatives by $\SP{k}$,
the formal definition is given in Section~\ref{sec:preliminaries})

The strategies yielded by the analysis guarantee bounds on the final
score. The adversarial strategy guarantees
$\lowerscore{} \leq \score^T$, while the learner's strategy
guarantees $\score^T \leq \upperscore{}$. Unfortunately, these bounds
don't match, i.e. $\lowerscore{}<\upperscore{}$. In other words our
proposed strategies are not min-max optimal. The question of whether
there exist min/max strategies for the integer time game is open.

To find min/max strategies we expand the game. We call the expanded
game the {\em discrete time game}. The expansion involves giving the
more options to the adversary, but not to the learner. As a result,
any upper bound $\upperscore{}$ that holds for a learner strategy in
the discrete time game also holds in the integer time game.

The added option for the adversary is to declare, at the beginning of
each iteration, the range of values of the instantanous losses. In the
integer time game this range is set to $[-1,+1]$. In the discrete time
game the range is chosen by the adversary on iteration $i$ to be
$[-s_i,+s_i]$ for $1\geq s_i >0$. To keep the game balanced between
the adversary and the learner we replace the iteration number $i$ with
real valued {\em time} parameter and let $t_{i+1}=t_i+s_i^2$. this and
another necessary adjustment are described in
Section~\ref{sec:discrete}. Section~\ref{sec:disc-game-strategies}
describes strategies used for the discrete game which are scaled
versions of the strategies for the integer time game.

We fix the potential at the end of the game $\finalPotR \in \SP{4}$
and consider a sequence of adversarial and learner strategies indexed
by $k$: $\adversMdk,\learnerMdk$, where
$\forall i,\; s^k_i = \frac{\sqrt{\realT}}{2^k}$ for some constant
$\realT$. We prove two facts regarding the limit $k \to \infty$.  The
first (Thm.~\ref{thm:seq-of-adv-strategies}) is that
$\lim_{k \to \infty}\upperpotMdk - \lowerpotMdk \to 0$.  The second
(Thm.~\ref{thm:smallerSteps}) is that, if $\pot(T,\R)$
$$\forall k, \forall 0\leq i \leq 2^{2k},t_i=i 2^{-2k} \realT,\; \forall\R,\;\;\;
,\lowerpotMdkpar{k+1}\left(t_i,R\right) >
\lowerpotMdk\left(t_i,R\right)$$ Taken together these facts imply
that, if the fixed potential function for the end of the game
$\finalPotR$ is in $\SP{4}$, then there exists a potential function
$\pot(t,\R)$. The adversarial strategy corresponding to this potential
function corresponds to Brownian motion.  The backwards recursion used
to computer the potential for $t \leq \realT$ is a partial
differential equation known as the Kolmogorov Backward equation.

The main result of this paper is that a {\em single} adversarial
strategy, i.e. Brownian motion, is optimal for any
sufficiently convex potential functions.

The discrete time game presents the adversary with a dilemma.  On the
one hand, The adversary has to declare, on each iteration, an upper bound on the
range of the losses $[-s_i,s_i]$ where $s_i>0$. On the other hand, it
wants to set $s_i$ as small as possible.~\footnote{The situation is
  similar to a folklore game in which each player writes down a number
  on a piece of paper and the player with the largest number wins.}

We introduce a variant of the game called the {\em continuous time
  game} to alleviate this dilemma. In this came the adversary does not
announce the step size and the learner behaves as if
the step size is infinitesimally small. In this case time is advanced according
to the variance of the actual losses. This much more natural algorithm yields
a regret bound that depends on the cumulative variance and is smaller
for easy, low variance sequences.

Until this point our theory holds for any final potential function in $\SP{4}$. We conclude by analyzing two specific potentials.
\begin{enumerate}
\item We derive a potential function and a corresponding learning algorithm
  that is min/max optimal for a given time horizon $\realT$.
  The optimality is in the sense that the
  simultanous regret bound for time $\realT$ has a matching
  simultanous lower bound.
\item By finding solutions to the Kolmogorov Backward equation that
  hold for all $t>0$ we eliminate the need to define a final
  potential. As a result we get an ``anytime'' learning algorithm that
  can be stopped at any time. The specific potential we analyze is
  Normal-Hedge~\cite{}. NormalHedge is not min/max optimal for any
  time, but it is almost optimal for all times.
\end{enumerate}




\section{Main Results}

\fi

%%% Local Variables:
%%% mode: latex
%%% TeX-master: t
%%% End:


\section{Upper and lower bounds on the simultanous regret}
Plan: (1) first order bound with Variable $\nu$ to show optimality.
(2) Second order bound with $\nu=1$

We now combine Theorems \ref{thm:simulBoundAveragePot} and~\ref{thm:variancebound} with
the Normal-Hedge potential~(\ref{eqn:NormalHedge}) to derive a second order bound on the regret of NormalHedge.
\begin{equation}
  \R_{\mbox{\tiny NH$(\nu)$}}(\epsilon) \leq \sqrt{(t_i+\nu) \left(2 \ln \frac{1}{2\epsilon}+
      \ln (t_i+\nu)\right)}
\end{equation}
Where $t_i = \nu+\sum_{j=1}^{i}\deltat_j$
\begin{equation}
  \deltat_i=
  \E{\R \sim \state(i)}{H(t_i,\R) \;\; \E{y \sim \adversM(t_i,\R)}{(y-\ell_i)^2}}
\end{equation}
and 
\begin{eqnarray}
  W(t_i,\R)&=& e^{\frac{R^2}{2t_i}} \left(\frac{1}{t^{3/2}} + \frac{R^2}{t^{5/2}} \right)\\
  H(t_i,\R)&=& \frac{W(t_i,R)}{\E{\rho \sim \state(i)}{W(t_i,\rho)}}
\end{eqnarray}

The initial potential is $1/\sqrt{\nu}$ and it remains this way in the continuous case. In the discrete case it is $1/\nu + O(1/n)$ where $n$ is the number of steps We can upper bound the potential by
$1/\nu+1$

\iffalse
\section{The concept of state and  a critique of quadratic variation}
Central to our analysis is the concept of the {\em state} of the DTOL
game described in figure~\ref{fig:DTOL}. A state is a function of
history that is minimally sufficient to compute the future of the
game.  Clearly, a sufficient state of the game at time $t$ is the
vector of $N$ cumulative regrets: $(R_1^t,\ldots,R_N^t)$. As the
indexing of the experts is immaterial we can replace the vector with
a distribution of regrets at time $t$ which we denote $\state(t)$.

This definition of state is sufficient in two senses. First, the
regret bounds at time $t$ depend only on $\state(t)$. Second, given
$\state(t)$ and the experts of the learner and the adversary at time
$t$ uniquely defines $\state(t)$. Other information about individual
experts, such as the quadratic variation of individual experts, do not
effect the progression of the states.

To sharpen this argument consider an example depicted in
Figure~\ref{fig:counterExample}~\footnote{This example appeared in an
  open problem in COLT2016} We compare two loss histories, each
involving two experts. The first consists of the the blue and the red
lines, the second consists of the blue and the green line. In both
cases the total loss of both expert at the end of phase 3 are
equal. In other words, the state at the end of phase 3 is identical
for the two examples. An algorithm that depends on quadratic variation
would give different weights to the high and the low variation expert
even though the states are identical.~\footnote{One can argue for a
  different game in which the state includes the quadratic
  variation. However, we can't see a natural Way to define a game that
  requires the expanded state.}
\begin{figure*}[th] \label{fig:counterExample}
\includegraphics[width=\columnwidth,height=3in]{OpenProblemFigure.pdf}
\end{figure*}
\fi
\section{Integer time game}
\label{sec:int-time-game}
The integer time game is described in
Figure~\ref{fig:integerTimeGame}.  The integer time game generalizes
the decision theoretic online learning problem~\cite{FreundSc97} in
the following ways:
\begin{enumerate}
\item The goal of the learner in DTOL is to guarantee an upper bounds
  on the regret. The learner's goal in the integer time game is to
  minimize the final score. From theorem~\ref{thm:simulBoundAveragePot} we know that if
  we set the final potential as $\finalPotT(\R) = \frac{1}{G(\R)}$ then the two
  conditions are equivalent, allowing us to focus on the score.
\item The number of iterations $T$ is given as input, as is the
  potential function at the end: $\finalPotT(\R)$.
\item The experts are assumed to be {\em divisible}. For our purposes
it is enough to assume that any expert can be split into two equal
weight parts.
\end{enumerate}

The key to the potential based analysis is that using the predefined
final potential we can define potential functions and scores for all
iterations $1,\ldots,T-1$. This is explained in the next subsection.

The final score is calculated: $\score^T=\state(T) \odot
\finalPotT$.

The goal of the learner is to minimize this score, the goal
of the adversary is to maximize it.
%%%%%%%%%%%%%%%%%%%%%%%%%%%%%%%%%%%%%%%%%%%%%%%%%%%%%%%%%%%%%%%
% \end{minipage}}                                             %
% \caption{The integer time game \label{fig:integerTimeGame}} %
% \end{figure}                                                %
%%%%%%%%%%%%%%%%%%%%%%%%%%%%%%%%%%%%%%%%%%%%%%%%%%%%%%%%%%%%%%%

\section{Proof of Theorem~\ref{thm:simulBoundAveragePot} \label{proof:simulBoundAveragePot}}
\begin{proof}
~\\
\begin{itemize}
  \item
  {\bf A distribution $\state$ satisfies SRB for $G$ if it satisfies APB
  for $\pot(R) = G(R)^{-1}$}\\
Assume by contradiction that $\state$ does not satisfy the simultaneous bound. In
other words there exists $a \in \reals$ such that
$\P{\R \sim \state}{\R > a} > B(a)$. From Markov inequality and the fact
that $\phi$ is non decreasing we get
\[
  \E{\R \sim \state}{\pot(\R)} \geq \phi(a) \P{\R \sim \state}{\R > a} >
  \phi(a) B(a) = \frac{B(a)}{B(a)}=1
\]
but $ \E{\R \sim \state}{\pot(\R)} >1$ contradicts the average potential
assumption for the potential $\phi(\R) = B(\R)^{-1}$
\item {\bf A distribution $\state$ satisfies an
    APB for $\pot$ if it satisfies SRB for $G$ and $\int_{-\infty}^{\infty}\pot(\R) G(\R) d\R \leq 1$}\\
  From the SRB condition we have
  $$\forall R, \P{\rho \sim \state}{\rho \geq \R} \leq G(\R)$$
  From the condition  $\int_{-\infty}^{\infty}\pot(\R) G(\R) d\R \leq 1$ we get
  $$1 \geq \int_{-\infty}^{\infty}\pot(\R) G(\R) d\R \geq
  \int_{-\infty}^{\infty}\pot(\R) \P{\rho \sim \state}{\rho\geq \R}
  d\R = \E{\R \sim \state}{\pot(\R)}$$
\end{itemize}
\end{proof}

\iffalse
memory-less adversaries. In other words
adversaries to not depend on history. In other words: for any $1 \leq i \leq T$:
\begin{eqnarray}
  && \P{\omega \sim \D}{L^i(\omega)=b_i \left|
  P^i(\omega)=a_i\wedge (P^{i-1}(\omega)=a_{i-1} \wedge L^{i-1}(\omega)=b_{i-1})
     \wedge \ldots \wedge (P^{1}(\omega)=a_{1} \wedge L^1(\omega)=b_1) \right.} \nonumber\\
     &=&   \P{\omega \sim \D}{L^i(\omega)=b_i}
\end{eqnarray}
\fi


\subsection{Potential-based analysis of DTOL}


Central to our analysis is the notion of a {\em game state}. From the game
state at the end of the game $i=T$ the outcome of the game can be calculated.



$i$, deonted $\state_i$ is a representation of the game at iteration $i$ from which (1) any quantity relevant to iteration $i$ can be calculated.



DTOL is a full information deterministicc game, if the strategies for
both sides are also deterministic then the game determines a single history:
\[
  \{P^1_1,P^1_2,\ldots,P^1_N\},\{l^1_1,l^1_2,\ldots,l_N^1\},
  \{P^2_1,P^2_2,\ldots,P^2_N\},\{l^2_1,l^2_2,\ldots,l_N^2\},\ldots 
\]
each player deterministically selects their expert according to a
prefix the history which is then appended to the end of the history
and the other side takes their turn.  It is not hard to verify that
the vector $\left\langle L^i_\ell,R_1^i,\ldots,R_N^i \right\rangle$
contains all of the information in the prefix of history up to
iteration $i$. Moreover, the information does not change if we relabel
the experts.  We therefor define the {\em state} of the game as
$\left\langle L^i_\ell, \{R_1^i,\ldots,R_N^i\} \right\rangle$ where the curly brackets define a {\em bag} or a {\em point-mass distribution}.

The first step in extending DTOL to EDTOL is allowing {\em any}
distribution over the regrets.

and also fix as well as
the final potential $\finalPotT$, we can calculate the conditional
expectation of the potential at time $T$ given the that the regret at
time $ti<T$ is R.  It is natural to define this conditional expected
value as the potential at $i,R$.





Central to our model is the concept of {\em state}. Roughly speaking,
the state of the game is a summary of the history which holds all of
the relevant past information. From Figure~\ref{fig:DTOL} it is clear
that the state on iteration $i$ can be defined as the combination of
the cumulative loss of the algorithm $L_\ell^i$ and the cumulative
regret of the $N$ experts:
$\left\langle L^i_\ell,R_1^i,\ldots,R_N^i \right\rangle$. As the
indices of the experts can be arbitrarily permuted, the cumulative
losses of the experts can be represented as a {\em bag} which can also
be thought of as a point-mass distribution over $\reals$. Here we
allow arbitrary distribution over the cumulative loss of the
expert.~\footnote{More precisely, an Borel-Meaure on $\reals$.} This
does not substrantially expand the setup with $N$ acctions as any
Borel-measure can approximated by letting $N \to \infty$. We use the
term {\em state} at time $i$, denoted $\state(i)$, to represent
distribution or measure, over regrets at iteration $i$.  The full
state at time $i$ is
$\fstate(i) \doteq \left\langle L^i_\ell,\state(i)
\right\rangle$.~\footnote{Note that according to our definitions the
  state is unchanged when exchanging the indices $i \neq j$ when
  $L_i^i=L_j^i$.  On the other hand, algorithms that depend on
  quadrative variation, depend also on a quadratic expression for each
  expert, which is problem with second order algorithms on quadratic
  variation, as these depend on information outside
  $\left\langle L^i_\ell,\state(t) \right\rangle$, namely, the
  quadratic variation of each expert.{\bf to be completed}}

Using this generalized definition of state we define a variant of the
DTOL game (Figure~\ref{fig:DTOL}) that we call the discrete time game (DTG)
(Figure~\ref{fig:discrete-Time-Game}). DTG generalizes DTOL and strengthens the adversary,
but not the learner. Thus any upper bound on the regret that holds for DTG holds also for DTOL. 

We now explain the elements of DGT in Figure~\ref{fig:discrete-Time-Game}
\begin{itemize}
\item the loss $l_j^i$ associated with expert is generalized to a measureble function that maps 
\item At the beginning of iteration $i$ the adversary chooses a
  step-size $1 \geq s_i \geq 0$ which sets the range of instatanous
  losss/gains to $[-s_i,+s_i]$
\item We replace the iteration number $i$ with $U_i$ that is an upper
  bound on the cumulative variance that is updated according to the recursion:
  $U_i = U_{i-1}+s_i^2$.
\item  
\end{itemize}


The DTOL game progresses from the full state $\fstate(t)$ to $\fstate(t+1)$ according to the strategies of the learner and the adversary:
\begin{enumerate}
\item The learner chooses a weighting $P(t,R)$ as a function of $\fstate(t)$
\item The adversary assigns a loss to each expert. $\adversM(t,\cdot): \reals \to \Delta^{[-s_i,+s_i]}$
  \item ,,
\end{enumerate}




$\left\langle L^i_\ell,\state(t) \right\rangle$ to
$\left\langle L^{t+1}_\ell,\state(t+1) \right\rangle$


a sequence of
states $s_0,s_1,s_2,\ldots,s_t$, and a strategy for the adversary
(potentially randomized) $A: S \times \Omega \to \vec{l}_t$, a
strategy for the learner $L:S,\Omega \to \vec{P}$.

such as DTOL described in figure~\ref{fig:DTOL}. 

\subsection{State symmetries}


Intuitively, the state of a
game captures all of the information about the game history that is
relevant to the game's future. More formally, a state is a function of
history that such that the future is independent of the history given
the state. One can define the state of the DTOL game at time $t$ as
the vector consisting of the $N$ cumulative regrets:
$(R_1^i,\ldots,R_N^i)$. Note also that the regret (relative to the
best expert or a percentile) as well as the potential are conserved
under permutation of the regrets. Therefor, a smaller definition of
the state is the {\em distribution} of the cumulative regrets.

When the set of experts is finite, this distribution is a point mass
distribution where the probability of a single point corresponds to
the number of experts with the cumulative regret.

We allow countable and uncountable sets of experts, however, for most of our analysis it sufficies to consider 
finite expert sets that double in size at each step.

Specifically we assume that the adversary chooses a step size $1 \leq s \leq 0$. At time $t$ the adversary
associates $2^t$ experts with the $2^t$ loss sequences of the form $\{\pm s\}^t$.

Thee set of ``divisible'' experts is somewhat larger. Here we allow
the adversary to split each expert at each stage into two, each expert
can have a different weight, but the weights have to sum that of the
expert before it is split.


%%%%%%%%%%%%%%%%%%%%%%%%%%%%%%%%%%%%%%%%

We define {\em the potential game} to be identical to DTOL with the
only difference being that the goal of the learner is to minimize the
final score and the goal of the adversary is to maximize it.

To simplify our initial analysis we assume that $T$, the number of
iterations in game (also called the {\em horizon} of the game) is
fixed and known a priori.  We will later remove this assumption.

We also assume that the final potential $\finalPotT$ is
fixed. Intuitively, we want $\finalPotT(R)$ to increase fast with $R$
so that $\alpha$ decreases quickly with $R$. However, at this point we
assume only that $\finalPotT(R)$ is in $\SP{2}$.

For our analysis we extend the game by giving the adversary additional
abilities which we shall refer to as {\em super powers}. These super
powers will simplify the analysis and allow us to prove min/max
optimality. As super-powers are given only to the adversary, any
learner strategy that achieves an upper bound on the final score in
the extended game will guarantee the same bound or better in the
original game (without super-powers).

Ultimately, we will show that the potential defined in the NormalHedge
algorithm is the {\em optimal} potential function that increases in
the fastest rate without resulting infinite expected values.

\paragraph*{full information game}
The potential game is a full information, deterministic, iterative,
finite horizon game between a learner and an adversary. The strategies
for the player and adverssary are mappings from the history of the
game: $\Pvec{1},\Lvec{1},\Pvec{2},\Lvec{2},\ldots$ to the next
choices: $\Pvec{i}$ for the learner and $\Lvec{i}$ for the learner.
The history implies a distribution over the regrets. In turn, the
distribution of the regrets determines the tail probability and the
average potential. There the distribution of the regret is the only
information we need to determine the result at the {\em end} of the
game. We will now show that the distribution of the regret at any time
point holds all of the relevant information about the history up to
that time point.

\paragraph*{The divisibility super-power}
The first super-power we give to the adversary is the ability to {\em
  split} experts. Suppose we have $N$ experts with non-negative
weights: $\omega_1,\ldots,\omega_N$ that sum to one.  The splitting
adversary can split any expert $j$ into $N_j$ ``sub-experts'' with
weights $\omega_{j,1},\ldots,\omega_{j,N_j}$ whose sum is
$\omega_j$.~\footnote{Thes $\omega_j$ are the a-priori or base weights
  associated with the experts, in particular, they do {\em not}
  incorporate the weighting by the learner $\Pvec{i}$}
Clearly, the non-splitting adversary is a special case of the
splitting adversary. Also, it is not hard to show that any splitting
adversary can be approximated as a non-splitting adversary by letting
$N \to \infty$ and using subsets of the experts to represent the
splitted expert.

\paragraph{The divisibility super-power allows the adversary to
  remove dependense on history}
As described above, the learner choses $\Pvec{i}$ at iteration $i$, 
as an arbitrary function of the history before iteration $i$
$\Pvec{1},\Lvec{1},\Pvec{i-1},\Lvec{i-1}$. Following this choice, the adversary chooses $\Lvec{i}$.
Let $\Rvec{i-1}$ be the distribution of the regrets before iteration $i$ and $\Rvec{i}$
be the distribution after iteration $i$. It is not hard to show that $\Rvec{i}$ is a function of
$\Rvec{i-1}, \Pvec{i}$ and $\Lvec{i}$. However, the full history might be used in the choices
made by the learner and the adversary.

We will not show that if the adversary has the divisibility super
power she can always replace $\Lvec{i}$ with $\Lvec{i}$ in a way that
the dependence on history reduces to a dependence on $\Rvec{i-1}$ and
$\Rvec{i}$ remains unchanged. Consider the finite experts case (the
general case holds under appropriate measurability constraints). Focus
on


\paragraph*{A Splitting adversary reduces history to distribution of experts}
Suppose that the regrets of the $N$ experts at iteration $T$ is
$\{R_1^T,\ldots,R_N^T\}$, and the corresponding weights are $\{\omega^T\}$

The final score, defined in Equation~(\ref{eqn:potential-bound}) is
only a function of the values .  As the
indices of the experts can be arbitrarily permuted, we can represent
the collection of values as a bag, or a point-mass distribution over
$\reals$. We will denote distribution by $\state(T)$, and for reasons justified in the following lemma, we call $\state(i)$ the {\em state} at iteration $i$. 

\begin{lemma}
  Assuming that the adversary has divisibility super power, For every
  iteration $1\leq i \leq T-1$, every history
  $\Pvec{1},\Lvec{1},\ldots,\Pvec{i}$ and any $\Lvec{i}$ resulting in
  distribution $\state(i+1)$ there is an alternative adversarial
  strategy $\Lvec{i'}$ that results in the same state $\state(i+1)$
  and in addition, $\state(i+1)$ is a function of
  $\state(i),\Pvec{i},\Lvec{i'}$
\end{lemma}

\begin{proof}
\end{proof}

This excludes second order alg based on quadratic variation.


The adversary in DTOL chooses a loss $l_j^i \in [-1,+1]$ for each expert $i$ at
iteration $j$ as a function of the game so far. We use a probabilistic
representation of the choice of the adversary, that is ..

%%% Local Variables:
%%% mode: latex
%%% TeX-master: t
%%% End:
